\chapter{Méthodologie Scrum et Gestion de Projet}

\section{Introduction}

La méthodologie Scrum constitue le pilier de notre approche de gestion de projet pour AWS Next Express. Cette méthodologie agile nous permet de gérer efficacement le développement logiciel en 3 semaines tout en maintenant une flexibilité face aux changements et en garantissant une livraison de valeur continue.

Ce chapitre détaille l'application concrète de Scrum dans notre projet court mais intensif, depuis l'organisation initiale jusqu'aux résultats obtenus. Nous présentons l'ensemble des processus et cérémonies Scrum adaptés à un contexte de 3 semaines.

\section{Fondamentaux Scrum dans le Projet}

\subsection{Introduction}

Cette section présente les fondamentaux de Scrum tels qu'appliqués dans notre projet AWS Next Express sur 3 semaines. Nous détaillons l'organisation de l'équipe, les rôles définis, et la structure générale de notre approche agile.

\subsection{Organisation de l'Équipe Scrum}

\begin{figure}[H]
    \centering
    \includegraphics[width=0.8\textwidth]{images/scrum_team_organization.png}
    \caption{Organisation de l'équipe Scrum}
    \label{fig:scrum_team}
\end{figure}

Notre équipe Scrum se compose de :

\begin{itemize}
    \item \textbf{Product Owner} : Nour el houda Bouajila - Gestion du backlog et des exigences
    \item \textbf{Scrum Master} : Ghofrane Nasri - Facilitation et coaching agile
    \item \textbf{Development Team} : Équipe auto-organisée de 2 développeurs
    \item \textbf{Stakeholders} : Encadrants académiques d'ITEAM University
\end{itemize}

\subsection{Framework Scrum Adapté}

\begin{figure}[H]
    \centering
    \includegraphics[width=1.0\textwidth]{images/scrum_framework_3weeks.png}
    \caption{Framework Scrum adapté pour 3 semaines}
    \label{fig:scrum_framework}
\end{figure}

\subsection{Conclusion}

L'organisation Scrum adoptée favorise la communication et la collaboration efficace dans un contexte temporel contraint de 3 semaines.

\section{Product Backlog et Gestion des Exigences}

\subsection{Introduction}

Le Product Backlog constitue le cœur de notre planification Scrum. Cette section présente l'organisation simple de notre backlog adapté à un projet de 3 semaines.

\subsection{Structure Simplifiée du Product Backlog}

\begin{figure}[H]
    \centering
    \includegraphics[width=1.0\textwidth]{images/simple_product_backlog.png}
    \caption{Structure simplifiée du Product Backlog}
    \label{fig:product_backlog}
\end{figure}

Notre Product Backlog s'organise en fonctionnalités prioritaires :

\begin{enumerate}
    \item \textbf{Interface utilisateur} : Dashboard et navigation
    \item \textbf{Gestion utilisateurs} : CRUD de base
    \item \textbf{Upload fichiers} : Intégration S3
    \item \textbf{Déploiement} : Mise en ligne
\end{enumerate}

\subsection{Priorisation Simple}

\begin{table}[H]
    \centering
    \begin{tabularx}{\textwidth}{|l|c|X|}
        \hline
        \textbf{Priorité} & \textbf{Fonctionnalités} & \textbf{Description} \\
        \hline
        Haute & 8 & Fonctionnalités essentielles (CRUD, UI) \\
        \hline
        Moyenne & 4 & Fonctionnalités importantes (Upload, validation) \\
        \hline
        Basse & 2 & Fonctionnalités bonus (Optimisations) \\
        \hline
    \end{tabularx}
    \caption{Répartition des fonctionnalités par priorité}
    \label{tab:priorities}
\end{table}

\subsection{Conclusion}

La gestion simplifiée du Product Backlog nous a permis de maintenir le focus sur l'essentiel en 3 semaines.

\section{Organisation des 3 Sprints}

\subsection{Introduction}

La planification des sprints constitue un élément clé de notre réussite. Cette section détaille notre approche de planification sur 3 sprints d'une semaine chacun.

\subsection{Vue d'Ensemble des 3 Sprints}

\begin{figure}[H]
    \centering
    \includegraphics[width=1.0\textwidth]{images/sprint_3weeks_timeline.png}
    \caption{Timeline des 3 sprints sur 3 semaines}
    \label{fig:sprint_timeline}
\end{figure}

\subsection{Capacité de l'Équipe}

\begin{table}[H]
    \centering
    \begin{tabularx}{\textwidth}{|c|c|c|X|}
        \hline
        \textbf{Sprint} & \textbf{Durée} & \textbf{Objectif} & \textbf{Livrables Principaux} \\
        \hline
        Sprint 1 & 1 semaine & Setup + UI & Projet configuré, interface de base \\
        \hline
        Sprint 2 & 1 semaine & Backend + CRUD & API fonctionnelle, DynamoDB intégré \\
        \hline
        Sprint 3 & 1 semaine & Finalisation & Upload S3, tests, déploiement \\
        \hline
    \end{tabularx}
    \caption{Organisation et objectifs des 3 sprints}
    \label{tab:sprint_organization}
\end{table>

\subsection{Conclusion}

La planification en 3 sprints nous a permis de maintenir un rythme de développement soutenu avec des objectifs clairs.

\section{Sprint 1 : Configuration et Interface}

\subsection{Introduction}

Le Sprint 1 pose les fondations du projet et établit l'interface utilisateur de base.

\subsection{Objectifs du Sprint 1}

\begin{figure}[H]
    \centering
    \includegraphics[width=0.9\textwidth]{images/sprint1_objectives_simple.png}
    \caption{Objectifs du Sprint 1}
    \label{fig:sprint1_objectives}
\end{figure}

\subsection{Réalisations Sprint 1}

\begin{itemize}
    \item Configuration du projet Next.js
    \item Setup des outils de développement
    \item Création des composants UI de base
    \item Interface responsive fonctionnelle
\end{itemize}

\subsection{Conclusion}

Le Sprint 1 a établi des bases solides pour le développement des fonctionnalités métier.

\section{Sprint 2 : Backend et Intégration AWS}

\subsection{Introduction}

Le Sprint 2 se concentre sur l'intégration du backend avec AWS et l'implémentation des fonctionnalités CRUD.

\subsection{Architecture Backend}

\begin{figure}[H]
    \centering
    \includegraphics[width=1.0\textwidth]{images/sprint2_backend_simple.png}
    \caption{Architecture backend Sprint 2}
    \label{fig:sprint2_backend}
\end{figure}

\subsection{Réalisations Sprint 2}

\begin{itemize}
    \item Intégration DynamoDB
    \item API Routes Next.js
    \item CRUD utilisateurs complet
    \item Validation des données
\end{itemize}

\subsection{Conclusion}

Le Sprint 2 a livré une application fonctionnelle avec persistance des données.

\section{Sprint 3 : Finalisation et Déploiement}

\subsection{Introduction}

Le Sprint 3 finalise l'application avec l'upload de fichiers, les tests et le déploiement.

\subsection{Finalisation}

\begin{figure}[H]
    \centering
    \includegraphics[width=1.0\textwidth]{images/sprint3_finalization.png}
    \caption{Finalisation Sprint 3}
    \label{fig:sprint3_final}
\end{figure}

\subsection{Réalisations Sprint 3}

\begin{itemize}
    \item Intégration S3 pour l'upload
    \item Tests essentiels
    \item Déploiement en production
    \item Documentation
\end{itemize}

\subsection{Conclusion}

Le Sprint 3 a livré une application complète et déployée.

\section{Cérémonies Scrum}

\subsection{Introduction}

Les cérémonies Scrum ont été adaptées à notre contexte de 3 semaines pour maintenir l'efficacité.

\subsection{Daily Standups}

\begin{figure}[H]
    \centering
    \includegraphics[width=0.8\textwidth]{images/daily_standups_simple.png}
    \caption{Organisation des Daily Standups}
    \label{fig:daily_standups}
\end{figure}

Format simple de nos Daily Standups :
\begin{itemize}
    \item \textbf{Durée} : 10 minutes maximum
    \item \textbf{Format} : Hier / Aujourd'hui / Blocages
    \item \textbf{Fréquence} : Quotidienne
\end{itemize}

\subsection{Sprint Reviews et Rétrospectives}

\begin{table}[H]
    \centering
    \begin{tabularx}{\textwidth}{|l|X|X|}
        \hline
        \textbf{Sprint} & \textbf{Points Positifs} & \textbf{Améliorations} \\
        \hline
        Sprint 1 & Setup efficace, UI moderne & Plus de tests unitaires \\
        \hline
        Sprint 2 & Intégration AWS réussie & Meilleure gestion des erreurs \\
        \hline
        Sprint 3 & Déploiement sans problème & Documentation plus détaillée \\
        \hline
    \end{tabularx}
    \caption{Synthèse des rétrospectives}
    \label{tab:retrospectives}
\end{table>

\subsection{Conclusion}

Les cérémonies adaptées ont maintenu la cohésion de l'équipe et l'efficacité du développement.

\section{Métriques et Performance}

\subsection{Introduction}

Le suivi des métriques nous permet d'évaluer l'efficacité de notre approche Scrum sur 3 semaines.

\subsection{Vélocité de l'Équipe}

\begin{figure}[H]
    \centering
    \includegraphics[width=1.0\textwidth]{images/velocity_3weeks.png}
    \caption{Vélocité sur 3 semaines}
    \label{fig:velocity}
\end{figure}

\subsection{Métriques Finales}

\begin{table}[H]
    \centering
    \begin{tabularx}{\textwidth}{|l|c|c|X|}
        \hline
        \textbf{Métrique} & \textbf{Objectif} & \textbf{Réalisé} & \textbf{Status} \\
        \hline
        Fonctionnalités livrées & 12 & 14 & ✓ Dépassé \\
        \hline
        Respect délais & 100\% & 100\% & ✓ Atteint \\
        \hline
        Qualité code & Bon & Très bon & ✓ Dépassé \\
        \hline
        Satisfaction équipe & 4/5 & 4.5/5 & ✓ Atteint \\
        \hline
    \end{tabularx}
    \caption{Métriques finales du projet}
    \label{tab:final_metrics}
\end{table>

\subsection{Conclusion}

Les métriques démontrent le succès de l'application Scrum sur 3 semaines.

\section{Conclusion du Chapitre}

Ce chapitre a démontré l'application réussie de la méthodologie Scrum dans un contexte de 3 semaines. Notre approche adaptative a permis de :

\subsection{Réalisations}

\begin{itemize}
    \item \textbf{Livraison rapide} : Application fonctionnelle en 3 semaines
    \item \textbf{Collaboration efficace} : Équipe coordonnée et productive
    \item \textbf{Qualité maintenue} : Standards respectés malgré les contraintes
    \item \textbf{Objectifs atteints} : Tous les objectifs principaux réalisés
\end{itemize}

\subsection{Leçons Apprises}

\begin{itemize}
    \item La simplification des processus Scrum est efficace pour les projets courts
    \item La communication quotidienne est essentielle dans un contexte serré
    \item La priorisation claire permet de se concentrer sur l'essentiel
    \item L'adaptation continue de la méthode améliore les résultats
\end{itemize}

L'application de Scrum sur 3 semaines valide cette méthodologie comme un framework efficace pour les projets académiques courts et intensifs. 