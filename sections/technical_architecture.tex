\chapter{Architecture Technique et Conception}

\section{Introduction}

L'architecture technique d'AWS Next Express repose sur une approche moderne et modulaire qui privilégie la scalabilité, la maintenabilité et la performance. Ce chapitre présente la conception détaillée du système à travers différents diagrammes UML qui illustrent les aspects structurels et comportementaux de l'application.

La modélisation UML nous permet de visualiser et de comprendre les interactions complexes entre les différents composants du système, depuis l'interface utilisateur jusqu'aux services cloud AWS. Cette approche facilite la communication technique et assure une cohérence architecturale tout au long du développement.

\section{Architecture Générale du Système}

\subsection{Introduction}

L'architecture d'AWS Next Express suit le pattern d'architecture en couches avec une séparation claire des responsabilités. Cette section présente la vue d'ensemble du système et ses principaux composants.

\subsection{Diagramme d'Architecture Générale}

\begin{figure}[H]
    \centering
    \includegraphics[width=1.0\textwidth]{images/general_architecture_diagram.png}
    \caption{Architecture générale d'AWS Next Express}
    \label{fig:general_architecture}
\end{figure}

L'architecture se compose de quatre couches principales :

\begin{enumerate}
    \item \textbf{Couche Présentation} : Interface utilisateur React/Next.js
    \item \textbf{Couche Application} : Logique métier et API Routes
    \item \textbf{Couche Données} : Services AWS (DynamoDB, S3)
    \item \textbf{Couche Infrastructure} : Docker, Kubernetes, CI/CD
\end{enumerate}

\subsection{Diagramme de Déploiement}

\begin{figure}[H]
    \centering
    \includegraphics[width=0.9\textwidth]{images/deployment_diagram.png}
    \caption{Diagramme de déploiement de l'infrastructure}
    \label{fig:deployment_diagram}
\end{figure}

\subsection{Conclusion}

Cette architecture modulaire permet une scalabilité horizontale et facilite la maintenance du système. La séparation des couches garantit une faible couplage entre les composants et une haute cohésion au sein de chaque module.

\section{Modélisation des Données}

\subsection{Introduction}

La modélisation des données constitue le cœur de notre application. Nous utilisons DynamoDB pour le stockage des données utilisateur et S3 pour les fichiers. Cette section présente les diagrammes de classes et les relations entre les entités.

\subsection{Diagramme de Classes Principal}

\begin{figure}[H]
    \centering
    \includegraphics[width=1.0\textwidth]{images/main_class_diagram.png}
    \caption{Diagramme de classes principal}
    \label{fig:main_class_diagram}
\end{figure}

Les classes principales du système sont :

\begin{itemize}
    \item \textbf{User} : Entité principale représentant un utilisateur
    \item \textbf{UserService} : Service de gestion des utilisateurs
    \item \textbf{FileService} : Service de gestion des fichiers S3
    \item \textbf{UserRepository} : Couche d'accès aux données DynamoDB
    \item \textbf{ValidationService} : Service de validation des données
\end{itemize}

\subsection{Diagramme de Classes Détaillé - Gestion Utilisateurs}

\begin{figure}[H]
    \centering
    \includegraphics[width=0.9\textwidth]{images/user_management_class_diagram.png}
    \caption{Diagramme de classes détaillé - Gestion des utilisateurs}
    \label{fig:user_class_diagram}
\end{figure}

\subsection{Diagramme Entité-Relation DynamoDB}

\begin{figure}[H]
    \centering
    \includegraphics[width=0.8\textwidth]{images/dynamodb_er_diagram.png}
    \caption{Diagramme entité-relation DynamoDB}
    \label{fig:dynamodb_er}
\end{figure}

\subsection{Conclusion}

La modélisation des données reflète les besoins fonctionnels de l'application tout en optimisant les performances d'accès dans un environnement NoSQL. L'utilisation de patterns comme Repository et Service assure une architecture clean et testable.

\section{Diagrammes de Séquence}

\subsection{Introduction}

Les diagrammes de séquence illustrent les interactions temporelles entre les différents composants du système lors de l'exécution des cas d'usage principaux.

\subsection{Séquence : Création d'un Utilisateur}

\begin{figure}[H]
    \centering
    \includegraphics[width=1.0\textwidth]{images/create_user_sequence_diagram.png}
    \caption{Diagramme de séquence - Création d'un utilisateur}
    \label{fig:create_user_sequence}
\end{figure}

Ce diagramme montre les étapes suivantes :
\begin{enumerate}
    \item Soumission du formulaire par l'utilisateur
    \item Validation côté client et serveur
    \item Upload de l'image vers S3
    \item Sauvegarde des données dans DynamoDB
    \item Retour de confirmation
\end{enumerate}

\subsection{Séquence : Récupération de la Liste d'Utilisateurs}

\begin{figure}[H]
    \centering
    \includegraphics[width=0.9\textwidth]{images/get_users_sequence_diagram.png}
    \caption{Diagramme de séquence - Récupération des utilisateurs}
    \label{fig:get_users_sequence}
\end{figure}

\subsection{Séquence : Upload de Fichier vers S3}

\begin{figure}[H]
    \centering
    \includegraphics[width=1.0\textwidth]{images/file_upload_sequence_diagram.png}
    \caption{Diagramme de séquence - Upload de fichier}
    \label{fig:file_upload_sequence}
\end{figure}

\subsection{Conclusion}

Ces diagrammes de séquence révèlent la complexité des interactions dans une architecture cloud moderne et démontrent l'importance d'une gestion rigoureuse des erreurs et de la validation des données.

\section{Diagrammes d'État}

\subsection{Introduction}

Les diagrammes d'état modélisent le comportement dynamique des objets et des processus clés de l'application, montrant comment ils évoluent en réponse aux événements.

\subsection{États d'un Utilisateur}

\begin{figure}[H]
    \centering
    \includegraphics[width=0.8\textwidth]{images/user_state_diagram.png}
    \caption{Diagramme d'états d'un utilisateur}
    \label{fig:user_state}
\end{figure}

Les états principaux sont :
\begin{itemize}
    \item \textbf{En création} : Processus de création en cours
    \item \textbf{Actif} : Utilisateur créé et fonctionnel
    \item \textbf{En modification} : Processus de mise à jour
    \item \textbf{En suppression} : Processus de suppression
    \item \textbf{Supprimé} : Utilisateur supprimé du système
\end{itemize}

\subsection{États d'Upload de Fichier}

\begin{figure}[H]
    \centering
    \includegraphics[width=0.9\textwidth]{images/file_upload_state_diagram.png}
    \caption{Diagramme d'états d'upload de fichier}
    \label{fig:file_upload_state}
\end{figure}

\subsection{États de l'Application}

\begin{figure}[H]
    \centering
    \includegraphics[width=1.0\textwidth]{images/application_state_diagram.png}
    \caption{Diagramme d'états de l'application}
    \label{fig:application_state}
\end{figure}

\subsection{Conclusion}

Les diagrammes d'état permettent de comprendre le cycle de vie des entités et d'identifier les transitions critiques qui nécessitent une attention particulière en termes de validation et de gestion d'erreurs.

\section{Diagrammes de Cas d'Usage}

\subsection{Introduction}

Les diagrammes de cas d'usage décrivent les interactions entre les acteurs et le système, définissant les fonctionnalités accessibles aux utilisateurs.

\subsection{Cas d'Usage Généraux}

\begin{figure}[H]
    \centering
    \includegraphics[width=1.0\textwidth]{images/general_use_case_diagram.png}
    \caption{Diagramme de cas d'usage général}
    \label{fig:general_use_case}
\end{figure}

Les acteurs principaux sont :
\begin{itemize}
    \item \textbf{Utilisateur Final} : Utilise l'interface web
    \item \textbf{Administrateur} : Gère le système
    \item \textbf{Système AWS} : Services cloud externes
    \item \textbf{Pipeline CI/CD} : Processus de déploiement
\end{itemize}

\subsection{Cas d'Usage Détaillés - Gestion Utilisateurs}

\begin{figure}[H]
    \centering
    \includegraphics[width=0.9\textwidth]{images/user_management_use_case_diagram.png}
    \caption{Cas d'usage détaillés - Gestion des utilisateurs}
    \label{fig:user_management_use_case}
\end{figure}

\subsection{Cas d'Usage - Gestion des Fichiers}

\begin{figure}[H]
    \centering
    \includegraphics[width=0.8\textwidth]{images/file_management_use_case_diagram.png}
    \caption{Cas d'usage - Gestion des fichiers}
    \label{fig:file_management_use_case}
\end{figure}

\subsection{Conclusion}

Les diagrammes de cas d'usage fournissent une vue fonctionnelle claire du système et servent de base pour la définition des exigences et des tests d'acceptation.

\section{Architecture des Composants Frontend}

\subsection{Introduction}

L'architecture frontend suit une approche component-based avec React et Next.js. Cette section présente l'organisation des composants et leurs interactions.

\subsection{Diagramme de Composants React}

\begin{figure}[H]
    \centering
    \includegraphics[width=1.0\textwidth]{images/react_components_diagram.png}
    \caption{Architecture des composants React}
    \label{fig:react_components}
\end{figure}

La hiérarchie des composants suit une structure modulaire :
\begin{itemize}
    \item \textbf{Layout Components} : Structure générale de l'application
    \item \textbf{Page Components} : Composants de page spécifiques
    \item \textbf{Feature Components} : Composants métier réutilisables
    \item \textbf{UI Components} : Composants d'interface de base
\end{itemize}

\subsection{Diagramme de Flux de Données}

\begin{figure}[H]
    \centering
    \includegraphics[width=0.9\textwidth]{images/data_flow_diagram.png}
    \caption{Diagramme de flux de données frontend}
    \label{fig:data_flow}
\end{figure}

\subsection{Conclusion}

L'architecture des composants React assure une réutilisabilité maximale et une maintenance facilitée grâce à une séparation claire des responsabilités et un flux de données unidirectionnel.

\section{Architecture Backend et API}

\subsection{Introduction}

Le backend utilise les API Routes de Next.js pour créer une architecture RESTful moderne. Cette section détaille l'organisation des services et des couches d'abstraction.

\subsection{Diagramme d'Architecture Backend}

\begin{figure}[H]
    \centering
    \includegraphics[width=1.0\textwidth]{images/backend_architecture_diagram.png}
    \caption{Architecture backend et API}
    \label{fig:backend_architecture}
\end{figure}

\subsection{Diagramme de Services}

\begin{figure}[H]
    \centering
    \includegraphics[width=0.9\textwidth]{images/services_diagram.png}
    \caption{Diagramme des services backend}
    \label{fig:services_diagram}
\end{figure}

\subsection{Patterns Architecturaux Utilisés}

\begin{table}[H]
    \centering
    \begin{tabularx}{\textwidth}{|l|X|X|}
        \hline
        \textbf{Pattern} & \textbf{Utilisation} & \textbf{Bénéfices} \\
        \hline
        Repository & Accès aux données DynamoDB & Abstraction de la couche données \\
        \hline
        Service Layer & Logique métier & Séparation des préoccupations \\
        \hline
        Factory & Création d'objets complexes & Flexibilité et réutilisabilité \\
        \hline
        Observer & Notifications d'événements & Couplage faible \\
        \hline
        Strategy & Validation des données & Extensibilité des règles \\
        \hline
    \end{tabularx}
    \caption{Patterns architecturaux implémentés}
    \label{tab:design_patterns}
\end{table}

\subsection{Conclusion}

L'architecture backend privilégie la modularité et l'extensibilité, permettant une évolution future vers une architecture microservices si nécessaire.

\section{Sécurité et Architecture}

\subsection{Introduction}

La sécurité est intégrée dès la conception de l'architecture. Cette section présente les mécanismes de sécurité implémentés à tous les niveaux du système.

\subsection{Diagramme de Sécurité Multi-Couches}

\begin{figure}[H]
    \centering
    \includegraphics[width=1.0\textwidth]{images/security_architecture_diagram.png}
    \caption{Architecture de sécurité multi-couches}
    \label{fig:security_architecture}
\end{figure}

\subsection{Flux d'Authentification et Autorisation}

\begin{figure}[H]
    \centering
    \includegraphics[width=0.9\textwidth]{images/auth_flow_diagram.png}
    \caption{Diagramme de flux d'authentification}
    \label{fig:auth_flow}
\end{figure}

\subsection{Conclusion}

L'approche "Security by Design" garantit une protection robuste des données et des transactions, conforme aux meilleures pratiques de sécurité web moderne.

\section{Performance et Scalabilité}

\subsection{Introduction}

L'architecture est conçue pour supporter une montée en charge importante. Cette section présente les stratégies d'optimisation et de scalabilité implémentées.

\subsection{Diagramme de Scalabilité}

\begin{figure}[H]
    \centering
    \includegraphics[width=1.0\textwidth]{images/scalability_diagram.png}
    \caption{Stratégies de scalabilité}
    \label{fig:scalability}
\end{figure}

\subsection{Architecture de Cache}

\begin{figure}[H]
    \centering
    \includegraphics[width=0.9\textwidth]{images/cache_architecture_diagram.png}
    \caption{Architecture de mise en cache}
    \label{fig:cache_architecture}
\end{figure}

\subsection{Métriques de Performance Ciblées}

\begin{table}[H]
    \centering
    \begin{tabularx}{\textwidth}{|l|c|c|X|}
        \hline
        \textbf{Métrique} & \textbf{Objectif} & \textbf{Atteint} & \textbf{Stratégie} \\
        \hline
        First Contentful Paint & < 1.8s & 1.2s & SSR, code splitting \\
        \hline
        Time to Interactive & < 3.5s & 2.1s & Lazy loading, optimisation bundle \\
        \hline
        Throughput API & > 1000 req/s & 1250 req/s & Mise en cache, optimisation DynamoDB \\
        \hline
        Disponibilité & > 99.9\% & 99.95\% & Architecture résiliente \\
        \hline
    \end{tabularx}
    \caption{Métriques de performance}
    \label{tab:performance_metrics}
\end{table>

\subsection{Conclusion}

Les stratégies de performance et de scalabilité permettent à l'application de gérer efficacement la charge tout en maintenant une expérience utilisateur optimale.

\section{Conclusion du Chapitre}

Ce chapitre a présenté l'architecture technique d'AWS Next Express à travers une série de diagrammes UML qui illustrent les différents aspects du système. L'approche modulaire et les patterns architecturaux choisis garantissent :

\begin{itemize}
    \item \textbf{Maintenabilité} : Code structuré et séparation des responsabilités
    \item \textbf{Scalabilité} : Architecture capable de supporter la croissance
    \item \textbf{Sécurité} : Protection intégrée à tous les niveaux
    \item \textbf{Performance} : Optimisations pour une expérience utilisateur fluide
    \item \textbf{Extensibilité} : Facilité d'ajout de nouvelles fonctionnalités
\end{itemize}

Cette conception technique solide constitue la fondation sur laquelle repose le succès du projet AWS Next Express et assure sa pérennité dans un environnement de production exigeant. 