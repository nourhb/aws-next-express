\chapter{Étude de projet}

\section{\faSearch\ Introduction}

Dans un contexte technologique en constante évolution, les entreprises cherchent à moderniser leurs infrastructures et leurs applications pour rester compétitives. Ce chapitre présente le cadre général du projet réalisé chez \textcolor{accentgreen}{\textbf{DigitalFromScratch}}, ainsi que les objectifs et la méthodologie adoptée pour le développement d'une application web full-stack moderne.

\section{\faBuilding\ Présentation de l'organisme d'accueil}

\subsection{DigitalFromScratch}

\begin{infobox}[À propos de DigitalFromScratch]
DigitalFromScratch est une entreprise spécialisée dans les solutions digitales innovantes, offrant des services de développement et d'implémentation pour les entreprises souhaitant moderniser leur présence numérique. L'entreprise se concentre sur l'adoption des meilleures pratiques en matière de développement web et d'infrastructure cloud.
\end{infobox}

\subsection{\faEye\ Mission et vision}

La mission de \textcolor{primaryblue}{\textbf{DigitalFromScratch}} est d'accompagner les entreprises dans leur transformation numérique en proposant des solutions modernes basées sur les technologies les plus récentes. L'entreprise vise à démocratiser l'accès aux technologies avancées et à promouvoir l'adoption de pratiques DevOps.

\subsection{\faCogs\ Domaines d'expertise}

L'entreprise couvre plusieurs domaines d'expertise :
\begin{itemize}
    \item \textcolor{primaryblue}{\textbf{Développement d'applications web modernes}} avec React et Next.js
    \item \textcolor{primaryblue}{\textbf{Architecture cloud}} avec AWS
    \item \textcolor{primaryblue}{\textbf{Implémentation de pipelines CI/CD}}
    \item \textcolor{primaryblue}{\textbf{Conteneurisation}} avec Docker et Kubernetes
    \item \textcolor{primaryblue}{\textbf{Infrastructure as Code}} avec Terraform
\end{itemize}

\section{\faExclamationTriangle\ Contexte du projet}

\subsection{Problématique}

Dans le paysage technologique actuel, de nombreuses entreprises font face à des défis majeurs :

\begin{warningbox}[Défis identifiés]
\begin{itemize}
    \item \textcolor{accentorange}{\textbf{Complexité technologique}} croissante
    \item \textcolor{accentorange}{\textbf{Besoin de scalabilité}} des applications
    \item \textcolor{accentorange}{\textbf{Exigences de sécurité}} renforcées
    \item \textcolor{accentorange}{\textbf{Pression pour réduire}} le time-to-market
    \item \textcolor{accentorange}{\textbf{Maintenance difficile}} des architectures traditionnelles
\end{itemize}
\end{warningbox}

\subsection{\faLightbulb\ Opportunités identifiées}

Face à ces défis, plusieurs opportunités se présentent :

\begin{itemize}
    \item \textcolor{accentgreen}{\textbf{Technologies modernes}} comme Next.js
    \item \textcolor{accentgreen}{\textbf{Infrastructure cloud AWS}} scalable
    \item \textcolor{accentgreen}{\textbf{Pratiques DevOps}} pour l'automatisation
    \item \textcolor{accentgreen}{\textbf{Conteneurisation}} pour la portabilité
    \item \textcolor{accentgreen}{\textbf{Infrastructure as Code}} pour la gestion
\end{itemize}

\section{\faTarget\ Objectifs du projet}

\subsection{Objectif principal}

\begin{infobox}[Objectif principal]
L'objectif principal de ce projet de \textbf{\color{primaryblue}3 semaines} est de développer et déployer une application web full-stack moderne qui démontre l'implémentation des meilleures pratiques actuelles en matière de développement web et d'architecture cloud.
\end{infobox}

\subsection{\faListOl\ Objectifs spécifiques}

Les objectifs spécifiques du projet incluent :

\begin{enumerate}
    \item \textcolor{primaryblue}{\textbf{Développer une interface utilisateur moderne}} avec Next.js et TypeScript
    \item \textcolor{primaryblue}{\textbf{Implémenter des API sécurisées}} avec Next.js API Routes
    \item \textcolor{primaryblue}{\textbf{Intégrer les services AWS}} pour le stockage et la base de données
    \item \textcolor{primaryblue}{\textbf{Mettre en place des pipelines CI/CD}} automatisés
    \item \textcolor{primaryblue}{\textbf{Packager l'application}} avec Docker
    \item \textcolor{primaryblue}{\textbf{Déployer sur une infrastructure cloud}}
    \item \textcolor{primaryblue}{\textbf{Implémenter une surveillance}} de l'application
    \item \textcolor{primaryblue}{\textbf{Appliquer les principes de sécurité}}
\end{enumerate}

\section{\faRocket\ Solution proposée}

\subsection{Vue d'ensemble}

\begin{infobox}[Architecture proposée]
La solution proposée consiste en une application web full-stack moderne utilisant une architecture cloud native. L'application sera développée avec Next.js, déployée sur AWS avec une infrastructure gérée par Terraform.
\end{infobox}

\subsection{\faLayerGroup\ Stack technologique}

La stack technologique sélectionnée comprend :

\begin{itemize}
    \item \textbf{\color{primaryblue}\faCode\ Frontend} : Next.js 14, TypeScript, Tailwind CSS
    \item \textbf{\color{primaryblue}\faServer\ Backend} : Next.js API Routes, NextAuth.js
    \item \textbf{\color{primaryblue}\faDatabase\ Base de données} : Amazon DynamoDB
    \item \textbf{\color{primaryblue}\faCloud\ Stockage} : Amazon S3
    \item \textbf{\color{primaryblue}\faDocker\ Conteneurisation} : Docker
    \item \textbf{\color{primaryblue}\faNetworkWired\ Orchestration} : Kubernetes
    \item \textbf{\color{primaryblue}\faCodeBranch\ CI/CD} : GitHub Actions
    \item \textbf{\color{primaryblue}\faTools\ Infrastructure} : Terraform
    \item \textbf{\color{primaryblue}\faChartLine\ Monitoring} : Amazon CloudWatch
\end{itemize}

\section{\faProjectDiagram\ Méthodologie de travail}

\subsection{\faAgile\ Approche Agile adaptée}

\begin{infobox}[Méthodologie]
Le projet adopte une méthodologie Agile adaptée au délai de 3 semaines, avec des sprints courts permettant une adaptation rapide et une livraison continue de valeur.
\end{infobox}

\subsection{\faCalendarWeek\ Phases du projet}

Le projet est structuré en trois phases principales sur \textbf{\color{primaryblue}3 semaines} :

\begin{enumerate}
    \item \textbf{\color{accentgreen}\faWeek\ Semaine 1 : Analyse et conception}
    \begin{itemize}
        \item Étude des besoins
        \item Analyse des technologies
        \item Conception de l'architecture
        \item Planification du projet
    \end{itemize}
    
    \item \textbf{\color{accentgreen}\faWeek\ Semaine 2 : Développement}
    \begin{itemize}
        \item Développement du frontend
        \item Implémentation des APIs
        \item Intégration des services AWS
        \item Développement des tests
    \end{itemize}
    
    \item \textbf{\color{accentgreen}\faWeek\ Semaine 3 : Déploiement et finalisation}
    \begin{itemize}
        \item Configuration de l'infrastructure
        \item Mise en place des pipelines CI/CD
        \item Déploiement et tests
        \item Documentation finale
    \end{itemize}
\end{enumerate}

\section{\faBoxes\ Livrables attendus}

\subsection{\faCode\ Livrables techniques}

Les livrables techniques du projet comprennent :

\begin{itemize}
    \item \textcolor{primaryblue}{\textbf{Application web full-stack}} fonctionnelle
    \item \textcolor{primaryblue}{\textbf{Code source documenté}} et versionné
    \item \textcolor{primaryblue}{\textbf{Infrastructure as Code}} avec Terraform
    \item \textcolor{primaryblue}{\textbf{Pipelines CI/CD}} configurés
    \item \textcolor{primaryblue}{\textbf{Images Docker}} optimisées
    \item \textcolor{primaryblue}{\textbf{Suite de tests}} automatisés
\end{itemize}

\subsection{\faFileAlt\ Livrables documentaires}

Les livrables documentaires incluent :

\begin{itemize}
    \item \textcolor{primaryblue}{\textbf{Rapport de projet}} de fin d'année
    \item \textcolor{primaryblue}{\textbf{Documentation d'architecture}}
    \item \textcolor{primaryblue}{\textbf{Guide de déploiement}}
    \item \textcolor{primaryblue}{\textbf{Documentation des APIs}}
    \item \textcolor{primaryblue}{\textbf{Manuel d'utilisation}}
\end{itemize}

\section{\faCheckCircle\ Conclusion}

Ce chapitre a présenté le cadre général du projet réalisé chez \textcolor{accentgreen}{\textbf{DigitalFromScratch}}, incluant la problématique identifiée, les objectifs fixés et la méthodologie adaptée au délai de \textbf{\color{primaryblue}3 semaines}. La solution proposée s'appuie sur des technologies modernes et des pratiques éprouvées pour répondre aux défis actuels du développement web.

\begin{warningbox}[Résumé du chapitre]
Le projet vise à créer une application de référence qui démontre l'efficacité des approches cloud native et DevOps. La méthodologie Agile adaptée permettra une livraison efficace dans les temps impartis.
\end{warningbox}

Le chapitre suivant présentera l'étude préalable, incluant l'analyse des technologies existantes et la justification des choix techniques effectués. 