\chapter{Annexes - Configurations Avancées}

\section{Configurations Infrastructure}

\subsection{Terraform Infrastructure}

\begin{lstlisting}[language=HCL, caption=terraform/main.tf]
terraform {
  required_version = ">= 1.0"
  required_providers {
    aws = {
      source  = "hashicorp/aws"
      version = "~> 5.0"
    }
  }
  
  backend "s3" {
    bucket = "aws-next-express-terraform-state"
    key    = "terraform.tfstate"
    region = "us-east-1"
  }
}

provider "aws" {
  region = var.aws_region
  
  default_tags {
    tags = {
      Project     = "aws-next-express"
      Environment = var.environment
      ManagedBy   = "terraform"
    }
  }
}

# Variables
variable "aws_region" {
  description = "AWS region"
  type        = string
  default     = "us-east-1"
}

variable "environment" {
  description = "Environment name"
  type        = string
  default     = "development"
}

variable "project_name" {
  description = "Project name"
  type        = string
  default     = "aws-next-express"
}

# DynamoDB Table
resource "aws_dynamodb_table" "users" {
  name           = "${var.project_name}-users-${var.environment}"
  billing_mode   = "PAY_PER_REQUEST"
  hash_key       = "id"
  
  attribute {
    name = "id"
    type = "S"
  }
  
  attribute {
    name = "email"
    type = "S"
  }
  
  global_secondary_index {
    name            = "EmailIndex"
    hash_key        = "email"
    projection_type = "ALL"
  }
  
  tags = {
    Name = "${var.project_name}-users-${var.environment}"
  }
}

# S3 Bucket
resource "aws_s3_bucket" "storage" {
  bucket = "${var.project_name}-storage-${var.environment}"
}

resource "aws_s3_bucket_public_access_block" "storage" {
  bucket = aws_s3_bucket.storage.id
  
  block_public_acls       = true
  block_public_policy     = true
  ignore_public_acls      = true
  restrict_public_buckets = true
}

resource "aws_s3_bucket_server_side_encryption_configuration" "storage" {
  bucket = aws_s3_bucket.storage.id
  
  rule {
    apply_server_side_encryption_by_default {
      sse_algorithm = "AES256"
    }
  }
}

# IAM Role for EKS
resource "aws_iam_role" "eks_node_group" {
  name = "${var.project_name}-eks-node-group-${var.environment}"
  
  assume_role_policy = jsonencode({
    Statement = [{
      Action = "sts:AssumeRole"
      Effect = "Allow"
      Principal = {
        Service = "ec2.amazonaws.com"
      }
    }]
    Version = "2012-10-17"
  })
}

# Outputs
output "dynamodb_table_name" {
  description = "Name of the DynamoDB table"
  value       = aws_dynamodb_table.users.name
}

output "s3_bucket_name" {
  description = "Name of the S3 bucket"
  value       = aws_s3_bucket.storage.bucket
}
\end{lstlisting}

\subsection{Configuration AWS CLI}

\begin{lstlisting}[language=bash, caption=aws-config.sh]
#!/bin/bash

# AWS CLI Configuration Script for AWS Next Express

set -e

echo "🔧 Configuring AWS CLI for AWS Next Express"
echo "============================================="

# Check if AWS CLI is installed
if ! command -v aws &> /dev/null; then
    echo "❌ AWS CLI is not installed. Please install it first."
    echo "📖 Installation guide: https://docs.aws.amazon.com/cli/latest/userguide/getting-started-install.html"
    exit 1
fi

# Check if AWS credentials are configured
if ! aws sts get-caller-identity &> /dev/null; then
    echo "🔑 AWS credentials not found. Starting configuration..."
    aws configure
else
    echo "✅ AWS credentials already configured"
    aws sts get-caller-identity
fi

# Set AWS region if not set
if [ -z "$AWS_REGION" ]; then
    export AWS_REGION="us-east-1"
    echo "🌍 AWS region set to: $AWS_REGION"
fi

# Create S3 bucket for Terraform state (if using Terraform)
TERRAFORM_BUCKET="aws-next-express-terraform-state"
if ! aws s3api head-bucket --bucket "$TERRAFORM_BUCKET" 2>/dev/null; then
    echo "📦 Creating Terraform state bucket: $TERRAFORM_BUCKET"
    aws s3api create-bucket --bucket "$TERRAFORM_BUCKET" --region "$AWS_REGION"
    aws s3api put-bucket-versioning --bucket "$TERRAFORM_BUCKET" --versioning-configuration Status=Enabled
    aws s3api put-bucket-encryption --bucket "$TERRAFORM_BUCKET" --server-side-encryption-configuration '{
        "Rules": [
            {
                "ApplyServerSideEncryptionByDefault": {
                    "SSEAlgorithm": "AES256"
                }
            }
        ]
    }'
fi

# Create application S3 bucket
APP_BUCKET="aws-next-express-storage"
if ! aws s3api head-bucket --bucket "$APP_BUCKET" 2>/dev/null; then
    echo "📦 Creating application storage bucket: $APP_BUCKET"
    aws s3api create-bucket --bucket "$APP_BUCKET" --region "$AWS_REGION"
    aws s3api put-bucket-encryption --bucket "$APP_BUCKET" --server-side-encryption-configuration '{
        "Rules": [
            {
                "ApplyServerSideEncryptionByDefault": {
                    "SSEAlgorithm": "AES256"
                }
            }
        ]
    }'
fi

# Create DynamoDB table
TABLE_NAME="users"
if ! aws dynamodb describe-table --table-name "$TABLE_NAME" &>/dev/null; then
    echo "🗄️  Creating DynamoDB table: $TABLE_NAME"
    aws dynamodb create-table \
        --table-name "$TABLE_NAME" \
        --attribute-definitions \
            AttributeName=id,AttributeType=S \
            AttributeName=email,AttributeType=S \
        --key-schema \
            AttributeName=id,KeyType=HASH \
        --global-secondary-indexes \
            IndexName=EmailIndex,KeySchema=[{AttributeName=email,KeyType=HASH}],Projection={ProjectionType=ALL},ProvisionedThroughput={ReadCapacityUnits=5,WriteCapacityUnits=5} \
        --provisioned-throughput ReadCapacityUnits=5,WriteCapacityUnits=5
    
    echo "⏳ Waiting for table to be active..."
    aws dynamodb wait table-exists --table-name "$TABLE_NAME"
fi

echo "✅ AWS configuration completed successfully!"
echo ""
echo "📝 Summary:"
echo "   - Region: $AWS_REGION"
echo "   - S3 Bucket (Terraform): $TERRAFORM_BUCKET"
echo "   - S3 Bucket (App): $APP_BUCKET"
echo "   - DynamoDB Table: $TABLE_NAME"
\end{lstlisting}

\section{Configuration GitHub Actions}

\subsection{Workflow de Test Avancé}

\begin{lstlisting}[language=YAML, caption=.github/workflows/test-pipeline.yml]
name: Test Pipeline

on:
  workflow_dispatch:
  schedule:
    - cron: '0 2 * * *' # Run daily at 2 AM

env:
  NODE_VERSION: '18'
  PNPM_VERSION: '8'

jobs:
  setup:
    runs-on: ubuntu-latest
    outputs:
      pnpm-cache: ${{ steps.pnpm-cache.outputs.cache-hit }}
    steps:
      - uses: actions/checkout@v4
      
      - name: Setup Node.js
        uses: actions/setup-node@v4
        with:
          node-version: ${{ env.NODE_VERSION }}
          
      - name: Setup pnpm
        uses: pnpm/action-setup@v2
        with:
          version: ${{ env.PNPM_VERSION }}
          
      - name: Get pnpm store directory
        id: pnpm-cache
        shell: bash
        run: |
          echo "STORE_PATH=$(pnpm store path)" >> $GITHUB_OUTPUT
          
      - name: Setup pnpm cache
        uses: actions/cache@v3
        with:
          path: ${{ steps.pnpm-cache.outputs.STORE_PATH }}
          key: ${{ runner.os }}-pnpm-store-${{ hashFiles('**/pnpm-lock.yaml') }}
          restore-keys: |
            ${{ runner.os }}-pnpm-store-

  lint-and-typecheck:
    runs-on: ubuntu-latest
    needs: setup
    steps:
      - uses: actions/checkout@v4
      
      - name: Setup Node.js and pnpm
        uses: ./.github/actions/setup-node-pnpm
        
      - name: Install dependencies
        run: pnpm install --frozen-lockfile
        
      - name: Lint
        run: pnpm run lint
        
      - name: Type check
        run: pnpm run type-check

  unit-tests:
    runs-on: ubuntu-latest
    needs: setup
    services:
      dynamodb:
        image: amazon/dynamodb-local:latest
        ports:
          - 8000:8000
        options: >-
          --health-cmd "curl -f http://localhost:8000"
          --health-interval 10s
          --health-timeout 5s
          --health-retries 5
    steps:
      - uses: actions/checkout@v4
      
      - name: Setup Node.js and pnpm
        uses: ./.github/actions/setup-node-pnpm
        
      - name: Install dependencies
        run: pnpm install --frozen-lockfile
        
      - name: Run unit tests
        run: pnpm run test:coverage
        env:
          DYNAMODB_ENDPOINT: http://localhost:8000
          AWS_ACCESS_KEY_ID: dummy
          AWS_SECRET_ACCESS_KEY: dummy
          AWS_REGION: us-east-1
          
      - name: Upload coverage to Codecov
        uses: codecov/codecov-action@v3
        with:
          file: ./coverage/lcov.info
          flags: unittests
          name: codecov-umbrella

  e2e-tests:
    runs-on: ubuntu-latest
    needs: setup
    services:
      dynamodb:
        image: amazon/dynamodb-local:latest
        ports:
          - 8000:8000
    steps:
      - uses: actions/checkout@v4
      
      - name: Setup Node.js and pnpm
        uses: ./.github/actions/setup-node-pnpm
        
      - name: Install dependencies
        run: pnpm install --frozen-lockfile
        
      - name: Install Playwright browsers
        run: pnpm exec playwright install --with-deps
        
      - name: Build application
        run: pnpm run build
        
      - name: Start application
        run: pnpm start &
        env:
          DYNAMODB_ENDPOINT: http://localhost:8000
          AWS_ACCESS_KEY_ID: dummy
          AWS_SECRET_ACCESS_KEY: dummy
          AWS_REGION: us-east-1
          
      - name: Wait for application
        run: npx wait-on http://localhost:3000
        
      - name: Run Playwright tests
        run: pnpm run test:e2e
        
      - name: Upload Playwright report
        uses: actions/upload-artifact@v3
        if: always()
        with:
          name: playwright-report
          path: playwright-report/
          retention-days: 30

  security-scan:
    runs-on: ubuntu-latest
    steps:
      - uses: actions/checkout@v4
      
      - name: Run Trivy vulnerability scanner
        uses: aquasecurity/trivy-action@master
        with:
          scan-type: 'fs'
          scan-ref: '.'
          format: 'sarif'
          output: 'trivy-results.sarif'
          
      - name: Upload Trivy scan results to GitHub Security tab
        uses: github/codeql-action/upload-sarif@v2
        with:
          sarif_file: 'trivy-results.sarif'

  performance-audit:
    runs-on: ubuntu-latest
    needs: setup
    steps:
      - uses: actions/checkout@v4
      
      - name: Setup Node.js and pnpm
        uses: ./.github/actions/setup-node-pnpm
        
      - name: Install dependencies
        run: pnpm install --frozen-lockfile
        
      - name: Build application
        run: pnpm run build
        
      - name: Start application
        run: pnpm start &
        
      - name: Wait for application
        run: npx wait-on http://localhost:3000
        
      - name: Run Lighthouse CI
        run: |
          npm install -g @lhci/cli@0.12.x
          lhci autorun
        env:
          LHCI_GITHUB_APP_TOKEN: ${{ secrets.LHCI_GITHUB_APP_TOKEN }}

  docker-test:
    runs-on: ubuntu-latest
    steps:
      - uses: actions/checkout@v4
      
      - name: Set up Docker Buildx
        uses: docker/setup-buildx-action@v3
        
      - name: Build Docker image
        uses: docker/build-push-action@v5
        with:
          context: .
          load: true
          tags: aws-next-express:test
          cache-from: type=gha
          cache-to: type=gha,mode=max
          
      - name: Test Docker image
        run: |
          docker run -d --name test-container -p 3000:3000 aws-next-express:test
          sleep 10
          curl -f http://localhost:3000/api/health || exit 1
          docker stop test-container
\end{lstlisting}

\section{Configuration ESLint et Prettier}

\subsection{Configuration ESLint Avancée}

\begin{lstlisting}[language=JavaScript, caption=eslint.config.js]
import { dirname } from "path";
import { fileURLToPath } from "url";
import { FlatCompat } from "@eslint/eslintrc";

const __filename = fileURLToPath(import.meta.url);
const __dirname = dirname(__filename);

const compat = new FlatCompat({
  baseDirectory: __dirname,
});

const eslintConfig = [
  ...compat.extends("next/core-web-vitals", "next/typescript"),
  {
    rules: {
      // TypeScript specific rules
      "@typescript-eslint/no-unused-vars": ["error", { 
        "argsIgnorePattern": "^_",
        "varsIgnorePattern": "^_" 
      }],
      "@typescript-eslint/no-explicit-any": "warn",
      "@typescript-eslint/prefer-const": "error",
      
      // React specific rules
      "react/no-unescaped-entities": "off",
      "react/display-name": "off",
      "react-hooks/rules-of-hooks": "error",
      "react-hooks/exhaustive-deps": "warn",
      
      // General rules
      "no-console": ["warn", { "allow": ["warn", "error"] }],
      "no-debugger": "error",
      "no-duplicate-imports": "error",
      "no-unused-expressions": "error",
      "prefer-const": "error",
      "no-var": "error",
      
      // Import rules
      "import/order": ["error", {
        "groups": [
          "builtin",
          "external",
          "internal",
          "parent",
          "sibling",
          "index"
        ],
        "pathGroups": [
          {
            "pattern": "@/**",
            "group": "internal",
            "position": "before"
          }
        ],
        "pathGroupsExcludedImportTypes": ["builtin"]
      }]
    }
  },
  {
    files: ["**/*.test.{ts,tsx}", "**/*.spec.{ts,tsx}"],
    rules: {
      "@typescript-eslint/no-explicit-any": "off",
      "no-console": "off"
    }
  }
];

export default eslintConfig;
\end{lstlisting}

\subsection{Configuration Prettier}

\begin{lstlisting}[language=JSON, caption=.prettierrc.json]
{
  "semi": true,
  "trailingComma": "es5",
  "singleQuote": true,
  "printWidth": 80,
  "tabWidth": 2,
  "useTabs": false,
  "bracketSpacing": true,
  "bracketSameLine": false,
  "arrowParens": "avoid",
  "endOfLine": "lf",
  "importOrder": [
    "^(react|next)(.*)$",
    "^@/(.*)$",
    "^[./]"
  ],
  "importOrderSeparation": true,
  "importOrderSortSpecifiers": true,
  "plugins": [
    "@trivago/prettier-plugin-sort-imports",
    "prettier-plugin-tailwindcss"
  ]
}
\end{lstlisting}

\section{Configuration Monitoring}

\subsection{Configuration Grafana}

\begin{lstlisting}[language=YAML, caption=monitoring/grafana-deployment.yaml]
apiVersion: apps/v1
kind: Deployment
metadata:
  name: grafana
  namespace: monitoring
spec:
  replicas: 1
  selector:
    matchLabels:
      app: grafana
  template:
    metadata:
      labels:
        app: grafana
    spec:
      containers:
      - name: grafana
        image: grafana/grafana:latest
        ports:
        - containerPort: 3000
        env:
        - name: GF_SECURITY_ADMIN_PASSWORD
          valueFrom:
            secretKeyRef:
              name: grafana-secret
              key: admin-password
        - name: GF_INSTALL_PLUGINS
          value: "grafana-kubernetes-app,grafana-clock-panel"
        volumeMounts:
        - name: grafana-storage
          mountPath: /var/lib/grafana
        - name: grafana-config
          mountPath: /etc/grafana/grafana.ini
          subPath: grafana.ini
        resources:
          requests:
            memory: "256Mi"
            cpu: "100m"
          limits:
            memory: "512Mi"
            cpu: "200m"
      volumes:
      - name: grafana-storage
        persistentVolumeClaim:
          claimName: grafana-pvc
      - name: grafana-config
        configMap:
          name: grafana-config
---
apiVersion: v1
kind: ConfigMap
metadata:
  name: grafana-config
  namespace: monitoring
data:
  grafana.ini: |
    [analytics]
    check_for_updates = true
    
    [grafana_net]
    url = https://grafana.net
    
    [log]
    mode = console
    
    [paths]
    data = /var/lib/grafana/
    logs = /var/log/grafana
    plugins = /var/lib/grafana/plugins
    provisioning = /etc/grafana/provisioning
    
    [server]
    protocol = http
    http_port = 3000
    domain = localhost
---
apiVersion: v1
kind: Service
metadata:
  name: grafana-service
  namespace: monitoring
spec:
  type: LoadBalancer
  ports:
  - port: 3000
    targetPort: 3000
  selector:
    app: grafana
\end{lstlisting}

\subsection{Alerting Rules}

\begin{lstlisting}[language=YAML, caption=monitoring/alerting-rules.yaml]
apiVersion: v1
kind: ConfigMap
metadata:
  name: prometheus-alerts
  namespace: monitoring
data:
  alerts.yml: |
    groups:
    - name: aws-next-express.rules
      rules:
      - alert: HighErrorRate
        expr: rate(http_requests_total{status_code=~"5.."}[5m]) > 0.1
        for: 5m
        labels:
          severity: critical
        annotations:
          summary: "High error rate detected"
          description: "Error rate is {{ $value }} errors per second"
          
      - alert: HighResponseTime
        expr: histogram_quantile(0.95, rate(http_request_duration_seconds_bucket[5m])) > 1
        for: 2m
        labels:
          severity: warning
        annotations:
          summary: "High response time detected"
          description: "95th percentile response time is {{ $value }}s"
          
      - alert: LowDiskSpace
        expr: node_filesystem_avail_bytes / node_filesystem_size_bytes * 100 < 10
        for: 5m
        labels:
          severity: warning
        annotations:
          summary: "Low disk space"
          description: "Disk usage is above 90%"
          
      - alert: HighMemoryUsage
        expr: (node_memory_MemTotal_bytes - node_memory_MemFree_bytes) / node_memory_MemTotal_bytes * 100 > 90
        for: 5m
        labels:
          severity: critical
        annotations:
          summary: "High memory usage"
          description: "Memory usage is {{ $value }}%"
          
      - alert: PodCrashLooping
        expr: rate(kube_pod_container_status_restarts_total[15m]) > 0
        for: 5m
        labels:
          severity: critical
        annotations:
          summary: "Pod is crash looping"
          description: "Pod {{ $labels.pod }} in namespace {{ $labels.namespace }} is restarting frequently"
\end{lstlisting}

Ces configurations avancées complètent l'infrastructure et les outils de développement d'AWS Next Express, offrant une solution robuste et professionnelle pour le déploiement et la maintenance de l'application. 