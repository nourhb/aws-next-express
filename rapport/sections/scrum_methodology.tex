\chapter{Méthodologie Scrum et Organisation du Projet}

\section{Introduction à Scrum}

\subsection{Principes Fondamentaux}

Scrum repose sur trois piliers fondamentaux :
\begin{enumerate}
    \item \textbf{Transparence} : Tous les aspects du processus doivent être visibles
    \item \textbf{Inspection} : Examen fréquent des artefacts et du progrès
    \item \textbf{Adaptation} : Ajustement rapide en cas de déviation
\end{enumerate}

Ces piliers sont soutenus par cinq valeurs Scrum :
\begin{itemize}
    \item \textbf{Engagement} : Dédication à atteindre les objectifs de l'équipe
    \item \textbf{Courage} : Faire ce qui est juste et travailler sur les problèmes difficiles
    \item \textbf{Focus} : Concentration sur le travail du Sprint et les objectifs
    \item \textbf{Ouverture} : Transparence sur le travail et les défis
    \item \textbf{Respect} : Respect mutuel entre les membres de l'équipe
\end{itemize}

\subsection{Framework Scrum}

\begin{figure}[H]
    \centering
    \includegraphics[width=1.0\textwidth]{images/scrum_framework.png}
    \caption{Framework Scrum complet}
    \label{fig:scrum_framework}
\end{figure}

Le framework Scrum se compose de :
\begin{itemize}
    \item \textbf{Rôles} : Product Owner, Scrum Master, Équipe de développement
    \item \textbf{Événements} : Sprint, Sprint Planning, Daily Scrum, Sprint Review, Sprint Retrospective
    \item \textbf{Artefacts} : Product Backlog, Sprint Backlog, Increment
\end{itemize}

\section{Organisation de l'Équipe}

\subsection{Composition de l'Équipe}

Notre équipe de développement était composée de :

\begin{table}[H]
    \centering
    \begin{tabularx}{\textwidth}{|l|X|X|}
        \hline
        \textbf{Membre} & \textbf{Rôle Principal} & \textbf{Responsabilités} \\
        \hline
        Nour el houda Bouajila & Développeuse Full-Stack / Scrum Master & Frontend, Backend, Infrastructure, Animation des cérémonies \\
        \hline
        Ghofrane Nasri & Développeuse Full-Stack / Product Owner & Backend, DevOps, Définition des besoins \\
        \hline
        Encadrant & Coach Agile & Guidance méthodologique, Validation des livrables \\
        \hline
    \end{tabularx}
    \caption{Composition et rôles de l'équipe}
    \label{tab:team_composition}
\end{table}

\subsection{Rôles et Responsabilités}

\subsubsection{Product Owner (Ghofrane)}
\begin{itemize}
    \item Définition et priorisation du Product Backlog
    \item Clarification des exigences et critères d'acceptation
    \item Validation des fonctionnalités développées
    \item Communication avec les parties prenantes
\end{itemize}

\subsubsection{Scrum Master (Nour el houda)}
\begin{itemize}
    \item Animation des cérémonies Scrum
    \item Facilitation et résolution des obstacles
    \item Protection de l'équipe des interruptions externes
    \item Coaching sur les pratiques Scrum
\end{itemize}

\subsubsection{Équipe de Développement (Toutes les deux)}
\begin{itemize}
    \item Développement des fonctionnalités
    \item Estimation et planification des tâches
    \item Tests et validation du code
    \item Collaboration et partage de connaissances
\end{itemize}

\section{Planification des Sprints}

\subsection{Durée et Calendrier}

Le projet s'est déroulé sur 12 semaines, organisées en 6 sprints de 2 semaines chacun :

\begin{table}[H]
    \centering
    \begin{tabularx}{\textwidth}{|c|X|X|X|}
        \hline
        \textbf{Sprint} & \textbf{Période} & \textbf{Objectif Principal} & \textbf{Livrables} \\
        \hline
        1 & Sem 1-2 & Setup et architecture & Configuration projet, conception architecture \\
        \hline
        2 & Sem 3-4 & Frontend de base & Interface utilisateur, composants React \\
        \hline
        3 & Sem 5-6 & Backend et APIs & API Routes, intégration DynamoDB \\
        \hline
        4 & Sem 7-8 & Fonctionnalités avancées & Upload S3, gestion utilisateurs complète \\
        \hline
        5 & Sem 9-10 & DevOps et tests & Docker, CI/CD, tests automatisés \\
        \hline
        6 & Sem 11-12 & Déploiement et documentation & Kubernetes, ArgoCD, documentation \\
        \hline
    \end{tabularx}
    \caption{Planning des sprints}
    \label{tab:sprint_planning}
\end{table}

\subsection{Gantt du Projet}

\begin{figure}[H]
    \centering
    \begin{ganttchart}[
        hgrid,
        vgrid,
        x unit=0.5cm,
        y unit title=0.8cm,
        y unit chart=0.7cm,
        bar/.append style={fill=primaryblue},
        milestone/.append style={fill=accentorange}
    ]{1}{12}
        \gantttitle{Semaines}{12} \\
        \gantttitlelist{1,...,12}{1} \\
        
        \ganttbar{Sprint 1: Setup}{1}{2} \\
        \ganttbar{Sprint 2: Frontend}{3}{4} \\
        \ganttbar{Sprint 3: Backend}{5}{6} \\
        \ganttbar{Sprint 4: Fonctionnalités}{7}{8} \\
        \ganttbar{Sprint 5: DevOps}{9}{10} \\
        \ganttbar{Sprint 6: Déploiement}{11}{12} \\
        
        \ganttmilestone{Demo Sprint 1}{2} \\
        \ganttmilestone{Demo Sprint 2}{4} \\
        \ganttmilestone{Demo Sprint 3}{6} \\
        \ganttmilestone{Demo Sprint 4}{8} \\
        \ganttmilestone{Demo Sprint 5}{10} \\
        \ganttmilestone{Livraison finale}{12}
    \end{ganttchart}
    \caption{Diagramme de Gantt du projet}
    \label{fig:project_gantt}
\end{figure}

\section{Product Backlog}

\subsection{Épiques et User Stories}

Le Product Backlog a été organisé autour de 5 épiques principales :

\begin{enumerate}
    \item \textbf{EP001 - Infrastructure de base}
    \begin{itemize}
        \item US001 : Configuration du projet Next.js avec TypeScript
        \item US002 : Setup de l'environnement de développement
        \item US003 : Configuration des outils de qualité de code
    \end{itemize}
    
    \item \textbf{EP002 - Interface utilisateur}
    \begin{itemize}
        \item US004 : Création du layout principal
        \item US005 : Développement du dashboard
        \item US006 : Interface de gestion des utilisateurs
        \item US007 : Formulaires de création/édition
    \end{itemize}
    
    \item \textbf{EP003 - Backend et API}
    \begin{itemize}
        \item US008 : Configuration DynamoDB
        \item US009 : API CRUD utilisateurs
        \item US010 : Intégration AWS S3
        \item US011 : Gestion des erreurs et validation
    \end{itemize}
    
    \item \textbf{EP004 - DevOps et déploiement}
    \begin{itemize}
        \item US012 : Containerisation Docker
        \item US013 : Configuration Kubernetes
        \item US014 : Pipeline CI/CD
        \item US015 : ArgoCD et déploiement continu
    \end{itemize}
    
    \item \textbf{EP005 - Tests et qualité}
    \begin{itemize}
        \item US016 : Tests unitaires
        \item US017 : Tests d'intégration
        \item US018 : Tests end-to-end
        \item US019 : Documentation technique
    \end{itemize}
\end{enumerate}

\subsection{Priorisation et Estimation}

\subsubsection{Critères de Priorisation}

La priorisation a été effectuée selon la méthode MoSCoW :
\begin{itemize}
    \item \textbf{Must have} : Fonctionnalités critiques pour le MVP
    \item \textbf{Should have} : Fonctionnalités importantes mais pas bloquantes
    \item \textbf{Could have} : Fonctionnalités désirables si le temps le permet
    \item \textbf{Won't have} : Fonctionnalités reportées à une version future
\end{itemize}

\subsubsection{Estimation en Story Points}

L'estimation a utilisé la suite de Fibonacci modifiée (1, 2, 3, 5, 8, 13, 21) :

\begin{table}[H]
    \centering
    \begin{tabularx}{\textwidth}{|X|c|c|X|}
        \hline
        \textbf{User Story} & \textbf{Priorité} & \textbf{Points} & \textbf{Justification} \\
        \hline
        US001 - Configuration Next.js & Must & 5 & Setup initial complexe \\
        \hline
        US004 - Layout principal & Must & 8 & Architecture UI fondamentale \\
        \hline
        US008 - Configuration DynamoDB & Must & 5 & Intégration AWS \\
        \hline
        US009 - API CRUD & Must & 13 & Logique métier complexe \\
        \hline
        US010 - Intégration S3 & Should & 8 & Upload de fichiers \\
        \hline
        US012 - Docker & Should & 5 & Configuration standard \\
        \hline
        US013 - Kubernetes & Could & 13 & Orchestration complexe \\
        \hline
        US014 - Pipeline CI/CD & Should & 8 & Automatisation DevOps \\
        \hline
    \end{tabularx}
    \caption{Estimation des User Stories principales}
    \label{tab:user_stories_estimation}
\end{table}

\section{Cérémonies Scrum}

\subsection{Sprint Planning}

\subsubsection{Objectifs et Déroulement}

Chaque Sprint Planning durait 2 heures et suivait cette structure :
\begin{enumerate}
    \item \textbf{Partie 1 (1h)} : Quoi faire ?
    \begin{itemize}
        \item Révision du Product Backlog
        \item Sélection des User Stories pour le Sprint
        \item Définition de l'objectif du Sprint
    \end{itemize}
    
    \item \textbf{Partie 2 (1h)} : Comment faire ?
    \begin{itemize}
        \item Décomposition en tâches techniques
        \item Estimation de l'effort
        \item Planification des dépendances
    \end{itemize}
\end{enumerate}

\subsubsection{Sprint Backlog}

Exemple du Sprint Backlog pour le Sprint 3 :

\begin{table}[H]
    \centering
    \begin{tabularx}{\textwidth}{|X|c|c|X|}
        \hline
        \textbf{Tâche} & \textbf{Assigné} & \textbf{Heures} & \textbf{Statut} \\
        \hline
        Configuration DynamoDB local & Ghofrane & 4 & Done \\
        \hline
        Modèle de données utilisateur & Nour & 2 & Done \\
        \hline
        API GET /users & Ghofrane & 3 & Done \\
        \hline
        API POST /users & Nour & 4 & Done \\
        \hline
        API PUT /users/\{id\} & Ghofrane & 3 & Done \\
        \hline
        API DELETE /users/\{id\} & Nour & 2 & Done \\
        \hline
        Tests unitaires APIs & Ghofrane & 6 & In Progress \\
        \hline
        Gestion d'erreurs & Nour & 3 & To Do \\
        \hline
    \end{tabularx}
    \caption{Exemple de Sprint Backlog (Sprint 3)}
    \label{tab:sprint3_backlog}
\end{table}

\subsection{Daily Scrum}

Les Daily Scrum se déroulaient tous les matins à 9h00 pendant 15 minutes :
\begin{itemize}
    \item \textbf{Qu'est-ce que j'ai fait hier ?}
    \item \textbf{Qu'est-ce que je vais faire aujourd'hui ?}
    \item \textbf{Quels obstacles ai-je rencontrés ?}
\end{itemize}

\subsection{Sprint Review}

Chaque Sprint Review incluait :
\begin{enumerate}
    \item Démonstration des fonctionnalités développées
    \item Retours du Product Owner et des parties prenantes
    \item Mise à jour du Product Backlog
    \item Discussion sur le prochain Sprint
\end{enumerate}

\subsection{Sprint Retrospective}

La rétrospective suivait le format "Start, Stop, Continue" :
\begin{itemize}
    \item \textbf{Start} : Nouvelles pratiques à adopter
    \item \textbf{Stop} : Pratiques à abandonner
    \item \textbf{Continue} : Pratiques à maintenir
\end{itemize}

\begin{figure}[H]
    \centering
    \includegraphics[width=0.8\textwidth]{images/retrospective_example.png}
    \caption{Exemple de tableau de rétrospective (Sprint 3)}
    \label{fig:retrospective_example}
\end{figure}

\section{Outils et Pratiques}

\subsection{Outils de Gestion de Projet}

\begin{table}[H]
    \centering
    \begin{tabularx}{\textwidth}{|l|X|X|}
        \hline
        \textbf{Outil} & \textbf{Usage} & \textbf{Avantages} \\
        \hline
        GitHub Projects & Kanban board, Sprint planning & Intégration native avec le code \\
        \hline
        Discord & Communication quotidienne & Channels organisés, historique \\
        \hline
        Google Meet & Daily Scrum, Cérémonies & Partage d'écran, enregistrement \\
        \hline
        Notion & Documentation, Rétrospectives & Collaboration temps réel \\
        \hline
    \end{tabularx}
    \caption{Outils utilisés pour la gestion de projet}
    \label{tab:project_tools}
\end{table}

\subsection{Définition de "Done"}

Une User Story est considérée comme "Done" quand :
\begin{enumerate}
    \item Le code est développé et testé
    \item Les tests unitaires passent (couverture > 80\%)
    \item Le code est revu par un pair
    \item La fonctionnalité est intégrée dans la branche main
    \item La documentation est mise à jour
    \item Le Product Owner valide la fonctionnalité
\end{enumerate}

\section{Métriques et Indicateurs}

\subsection{Burndown Chart}

\begin{figure}[H]
    \centering
    \includegraphics[width=0.9\textwidth]{images/burndown_chart.png}
    \caption{Burndown chart du Sprint 4}
    \label{fig:burndown_chart}
\end{figure}

\subsection{Vélocité de l'Équipe}

\begin{table}[H]
    \centering
    \begin{tabularx}{\textwidth}{|c|c|c|c|c|}
        \hline
        \textbf{Sprint} & \textbf{Points Planifiés} & \textbf{Points Réalisés} & \textbf{Vélocité} & \textbf{Commentaires} \\
        \hline
        1 & 15 & 12 & 80\% & Apprentissage des outils \\
        \hline
        2 & 18 & 18 & 100\% & Rythme stabilisé \\
        \hline
        3 & 20 & 19 & 95\% & Complexité DynamoDB \\
        \hline
        4 & 20 & 21 & 105\% & Bonne productivité \\
        \hline
        5 & 22 & 20 & 91\% & Défis DevOps \\
        \hline
        6 & 18 & 18 & 100\% & Finalisation réussie \\
        \hline
    \end{tabularx}
    \caption{Évolution de la vélocité par sprint}
    \label{tab:velocity_evolution}
\end{table}

\section{Défis Rencontrés et Solutions}

\subsection{Défis Techniques}

\begin{enumerate}
    \item \textbf{Intégration DynamoDB}
    \begin{itemize}
        \item \textit{Problème} : Configuration complexe des indexes
        \item \textit{Solution} : Documentation AWS et prototypage
    \end{itemize}
    
    \item \textbf{Configuration Kubernetes}
    \begin{itemize}
        \item \textit{Problème} : Gestion des secrets et ConfigMaps
        \item \textit{Solution} : Formation en ligne et mentoring
    \end{itemize}
\end{enumerate}

\subsection{Défis Organisationnels}

\begin{enumerate}
    \item \textbf{Coordination à distance}
    \begin{itemize}
        \item \textit{Problème} : Synchronisation des tâches
        \item \textit{Solution} : Daily Scrum rigoureux et outils collaboratifs
    \end{itemize}
    
    \item \textbf{Gestion du temps}
    \begin{itemize}
        \item \textit{Problème} : Sous-estimation de certaines tâches
        \item \textit{Solution} : Amélioration continue des estimations
    \end{itemize}
\end{enumerate}

\section{Leçons Apprises}

\subsection{Bénéfices de Scrum}

\begin{itemize}
    \item \textbf{Visibilité} : Transparence constante sur l'avancement
    \item \textbf{Adaptabilité} : Ajustements rapides aux changements
    \item \textbf{Qualité} : Reviews fréquentes et feedback continu
    \item \textbf{Motivation} : Objectifs clairs et livraisons régulières
\end{itemize}

\subsection{Améliorations Possibles}

\begin{itemize}
    \item Utilisation d'outils de burn-down automatisés
    \item Intégration plus poussée des métriques de qualité
    \item Formation approfondie sur l'estimation
    \item Amélioration de la définition des critères d'acceptation
\end{itemize}

Cette expérience avec Scrum nous a permis de livrer un projet complexe de manière structurée et efficace, tout en maintenant une qualité élevée et en s'adaptant aux défis rencontrés. 