\chapter{Introduction}

\section{Contexte Général}

Dans un monde numérique en constante évolution, le développement d'applications web modernes nécessite une approche méthodique et des technologies robustes. Les entreprises recherchent aujourd'hui des solutions full-stack capables de gérer des volumes de données importants tout en offrant une expérience utilisateur optimale et une scalabilité horizontale.

Ce rapport présente le développement d'AWS Next Express, une application web complète qui illustre l'implémentation d'une architecture moderne basée sur les technologies cloud et les meilleures pratiques DevOps. Le projet a été conçu comme une plateforme de gestion d'utilisateurs avec upload de fichiers, intégrant les services AWS et déployée sur une infrastructure Kubernetes.

\section{Problématique}

Les défis actuels du développement web moderne incluent :

\begin{itemize}
    \item \textbf{Scalabilité} : Gérer l'augmentation du trafic et des données
    \item \textbf{Performance} : Optimiser les temps de réponse et l'expérience utilisateur
    \item \textbf{Sécurité} : Protéger les données sensibles et prévenir les vulnérabilités
    \item \textbf{Maintenance} : Faciliter les mises à jour et la maintenance du code
    \item \textbf{Déploiement} : Automatiser les processus de livraison continue
    \item \textbf{Collaboration} : Permettre un travail d'équipe efficace avec des méthodologies agiles
\end{itemize}

Notre projet répond à ces défis en proposant une solution complète utilisant :
\begin{itemize}
    \item Un framework frontend moderne (Next.js 15)
    \item Une base de données NoSQL scalable (DynamoDB)
    \item Un stockage cloud sécurisé (AWS S3)
    \item Une containerisation avec Docker
    \item Une orchestration Kubernetes
    \item Un pipeline CI/CD automatisé
\end{itemize}

\section{Objectifs du Projet}

\subsection{Objectif Principal}

Développer une application web full-stack moderne en utilisant la méthodologie Scrum, intégrant les meilleures pratiques du développement logiciel et du DevOps.

\subsection{Objectifs Spécifiques}

\begin{enumerate}
    \item \textbf{Développement Frontend} :
    \begin{itemize}
        \item Créer une interface utilisateur moderne avec Next.js 15
        \item Implémenter un design responsive avec Tailwind CSS
        \item Développer des composants réutilisables avec TypeScript
        \item Intégrer un système de gestion d'état efficace
    \end{itemize}
    
    \item \textbf{Développement Backend} :
    \begin{itemize}
        \item Concevoir des API RESTful avec Next.js API Routes
        \item Intégrer DynamoDB pour la persistance des données
        \item Implémenter l'upload de fichiers avec AWS S3
        \item Assurer la sécurité et la validation des données
    \end{itemize}
    
    \item \textbf{Infrastructure et DevOps} :
    \begin{itemize}
        \item Containeriser l'application avec Docker
        \item Orchestrer les services avec Kubernetes
        \item Mettre en place un pipeline CI/CD
        \item Implémenter le déploiement continu avec ArgoCD
    \end{itemize}
    
    \item \textbf{Méthodologie et Qualité} :
    \begin{itemize}
        \item Appliquer la méthodologie Scrum
        \item Développer une suite de tests complète
        \item Implémenter des pre-commit hooks
        \item Documenter le projet de manière exhaustive
    \end{itemize}
\end{enumerate}

\section{Méthodologie Adoptée}

Notre approche s'articule autour de la méthodologie Scrum, qui favorise :
\begin{itemize}
    \item Le développement itératif et incrémental
    \item La collaboration d'équipe
    \item L'adaptation aux changements
    \item La livraison de valeur continue
\end{itemize}

Le projet a été organisé en sprints de 2 semaines sur une période totale de 12 semaines, permettant une progression structurée et des points de contrôle réguliers.

\section{Structure du Rapport}

Ce rapport est organisé en plusieurs chapitres :

\begin{itemize}
    \item \textbf{Chapitre 2} : Présentation du contexte et de l'état de l'art
    \item \textbf{Chapitre 3} : Méthodologie Scrum et organisation du projet
    \item \textbf{Chapitre 4} : Architecture technique et conception
    \item \textbf{Chapitre 5} : Implémentation et développement
    \item \textbf{Chapitre 6} : Tests et validation
    \item \textbf{Chapitre 7} : DevOps et déploiement
    \item \textbf{Chapitre 8} : Résultats et évaluation
    \item \textbf{Chapitre 9} : Conclusion et perspectives
\end{itemize}

\section{Contributions}

Ce projet apporte plusieurs contributions significatives :

\begin{enumerate}
    \item \textbf{Technique} : Démonstration d'une architecture full-stack moderne avec intégration cloud
    \item \textbf{Méthodologique} : Application pratique de Scrum dans un projet de développement logiciel
    \item \textbf{Pédagogique} : Documentation complète pour servir de référence à d'autres projets
    \item \textbf{Pratique} : Solution prête pour la production avec pipeline CI/CD complet
\end{enumerate}

\begin{figure}[H]
    \centering
    \includegraphics[width=0.8\textwidth]{images/project_overview.png}
    \caption{Vue d'ensemble du projet AWS Next Express}
    \label{fig:project_overview}
\end{figure} 