\chapter{Annexes - Code et Configurations}

\section{Configurations Projet}

\subsection{Package.json Principal}

\begin{lstlisting}[language=JSON, caption=package.json]
{
  "name": "aws-next-express",
  "version": "1.0.0",
  "private": true,
  "scripts": {
    "dev": "next dev",
    "build": "next build",
    "start": "next start",
    "lint": "next lint",
    "test": "jest",
    "test:watch": "jest --watch",
    "test:coverage": "jest --coverage",
    "test:e2e": "playwright test",
    "test:e2e:ui": "playwright test --ui",
    "type-check": "tsc --noEmit",
    "docker:build": "docker build -t aws-next-express .",
    "docker:run": "docker run -p 3000:3000 aws-next-express",
    "docker:compose": "docker-compose up -d",
    "k8s:deploy": "kubectl apply -f k8s/",
    "k8s:delete": "kubectl delete -f k8s/"
  },
  "dependencies": {
    "@aws-sdk/client-dynamodb": "^3.621.0",
    "@aws-sdk/client-s3": "^3.621.0",
    "@aws-sdk/lib-dynamodb": "^3.621.0",
    "@radix-ui/react-dialog": "^1.1.1",
    "@radix-ui/react-dropdown-menu": "^2.1.1",
    "@radix-ui/react-label": "^2.1.0",
    "@radix-ui/react-slot": "^1.1.0",
    "class-variance-authority": "^0.7.0",
    "clsx": "^2.1.1",
    "lucide-react": "^0.445.0",
    "next": "15.2.4",
    "react": "^18.3.1",
    "react-dom": "^18.3.1",
    "tailwind-merge": "^2.5.4",
    "tailwindcss-animate": "^1.0.7",
    "zod": "^3.23.8"
  },
  "devDependencies": {
    "@next/eslint-config-next": "15.2.4",
    "@playwright/test": "^1.47.2",
    "@testing-library/jest-dom": "^6.5.0",
    "@testing-library/react": "^16.0.1",
    "@testing-library/user-event": "^14.5.2",
    "@types/jest": "^29.5.14",
    "@types/node": "^20.16.11",
    "@types/react": "^18.3.12",
    "@types/react-dom": "^18.3.1",
    "eslint": "^8.57.1",
    "jest": "^29.7.0",
    "jest-environment-jsdom": "^29.7.0",
    "postcss": "^8.4.49",
    "tailwindcss": "^3.4.17",
    "typescript": "^5.8.3"
  }
}
\end{lstlisting}

\subsection{Variables d'Environnement}

\begin{lstlisting}[language=bash, caption=env.example]
# AWS Configuration
AWS_REGION=us-east-1
AWS_ACCESS_KEY_ID=your_access_key_here
AWS_SECRET_ACCESS_KEY=your_secret_key_here
AWS_S3_BUCKET_NAME=aws-next-express-bucket

# DynamoDB Configuration
DYNAMODB_USERS_TABLE=users
DYNAMODB_ENDPOINT=http://localhost:8000

# Application Configuration
NODE_ENV=development
NEXT_PUBLIC_APP_URL=http://localhost:3000

# Security
CSRF_SECRET=your_csrf_secret_here
JWT_SECRET=your_jwt_secret_here

# Monitoring
PROMETHEUS_ENABLED=true
LOG_LEVEL=info
\end{lstlisting}

\section{Scripts d'Automatisation}

\subsection{Script de Configuration Initiale}

\begin{lstlisting}[language=bash, caption=scripts/setup-github.sh]
#!/bin/bash

# Colors for output
RED='\033[0;31m'
GREEN='\033[0;32m'
YELLOW='\033[1;33m'
NC='\033[0m' # No Color

echo -e "${GREEN}🚀 Setting up AWS Next Express GitHub Repository${NC}"

# Check if git is initialized
if [ ! -d ".git" ]; then
    echo -e "${YELLOW}Initializing git repository...${NC}"
    git init
fi

# Create .gitignore if it doesn't exist
if [ ! -f ".gitignore" ]; then
    echo -e "${YELLOW}Creating .gitignore...${NC}"
    cat > .gitignore << EOF
# Dependencies
node_modules/
.pnpm-store/

# Next.js
.next/
out/

# Environment variables
.env.local
.env.development.local
.env.test.local
.env.production.local

# Logs
npm-debug.log*
yarn-debug.log*
yarn-error.log*
.pnpm-debug.log*

# Runtime data
pids
*.pid
*.seed
*.pid.lock

# Coverage directory used by tools like istanbul
coverage/
*.lcov

# Dependency directories
node_modules/

# Optional npm cache directory
.npm

# Optional REPL history
.node_repl_history

# Output of 'npm pack'
*.tgz

# Yarn Integrity file
.yarn-integrity

# dotenv environment variables file
.env

# Mac
.DS_Store

# Windows
Thumbs.db

# IDE
.vscode/
.idea/
*.swp
*.swo

# Testing
/coverage
/test-results/
/playwright-report/
/playwright/.cache/

# Docker
.dockerignore
EOF
fi

# Add all files
echo -e "${YELLOW}Adding files to git...${NC}"
git add .

# Initial commit
echo -e "${YELLOW}Creating initial commit...${NC}"
git commit -m "Initial commit: AWS Next Express project setup

- Next.js 15 with TypeScript and Tailwind CSS
- DynamoDB integration with AWS SDK
- Docker containerization
- Kubernetes deployment manifests
- GitHub Actions CI/CD pipeline
- Comprehensive testing setup
- Documentation and README"

# Check if origin remote exists
if ! git remote get-url origin > /dev/null 2>&1; then
    echo -e "${YELLOW}Adding GitHub remote...${NC}"
    git remote add origin https://github.com/nourhb/aws-next-express.git
fi

# Push to GitHub
echo -e "${YELLOW}Pushing to GitHub...${NC}"
git branch -M main
git push -u origin main

echo -e "${GREEN}✅ Repository successfully set up and pushed to GitHub!${NC}"
echo -e "${GREEN}🔗 Repository URL: https://github.com/nourhb/aws-next-express${NC}"
\end{lstlisting}

\subsection{Script de Démarrage Rapide}

\begin{lstlisting}[language=bash, caption=scripts/quick-start.sh]
#!/bin/bash

set -e

echo "🚀 AWS Next Express Quick Start"
echo "================================"

# Check dependencies
command -v node >/dev/null 2>&1 || { echo "❌ Node.js is required but not installed. Aborting." >&2; exit 1; }
command -v docker >/dev/null 2>&1 || { echo "❌ Docker is required but not installed. Aborting." >&2; exit 1; }

echo "✅ Dependencies check passed"

# Install pnpm if not available
if ! command -v pnpm &> /dev/null; then
    echo "📦 Installing pnpm..."
    npm install -g pnpm
fi

# Install dependencies
echo "📦 Installing project dependencies..."
pnpm install

# Copy environment file
if [ ! -f ".env.local" ]; then
    echo "🔧 Setting up environment variables..."
    cp env.example .env.local
    echo "⚠️  Please edit .env.local with your AWS credentials"
fi

# Start DynamoDB Local
echo "🗄️  Starting DynamoDB Local..."
docker run -d \
  --name dynamodb-local \
  -p 8000:8000 \
  amazon/dynamodb-local:latest \
  -jar DynamoDBLocal.jar -sharedDb

# Wait for DynamoDB to be ready
echo "⏳ Waiting for DynamoDB to be ready..."
sleep 5

# Initialize DynamoDB tables
echo "🏗️  Creating DynamoDB tables..."
node scripts/init-dynamodb.js

# Run tests
echo "🧪 Running tests..."
pnpm test

# Build application
echo "🏗️  Building application..."
pnpm build

echo "✅ Quick start completed successfully!"
echo ""
echo "🚀 To start development:"
echo "   pnpm dev"
echo ""
echo "🐳 To start with Docker:"
echo "   pnpm docker:compose"
echo ""
echo "🌐 Application will be available at:"
echo "   http://localhost:3000"
\end{lstlisting}

\section{Utilitaires de Test}

\subsection{Configuration Jest Complète}

\begin{lstlisting}[language=TypeScript, caption=jest.config.ts]
import type { Config } from 'jest'
import nextJest from 'next/jest'

const createJestConfig = nextJest({
  // Provide the path to your Next.js app to load next.config.js and .env files
  dir: './',
})

// Add any custom config to be passed to Jest
const config: Config = {
  coverageProvider: 'v8',
  testEnvironment: 'jsdom',
  setupFilesAfterEnv: ['<rootDir>/jest.setup.js'],
  moduleNameMapper: {
    // Handle module aliases (this will be automatically configured for you based on your tsconfig.json paths)
    '^@/(.*)$': '<rootDir>/$1',
    '^@/components/(.*)$': '<rootDir>/components/$1',
    '^@/lib/(.*)$': '<rootDir>/lib/$1',
    '^@/app/(.*)$': '<rootDir>/app/$1',
    '^@/hooks/(.*)$': '<rootDir>/hooks/$1',
  },
  testMatch: [
    '**/__tests__/**/*.(ts|tsx|js)',
    '**/*.(test|spec).(ts|tsx|js)'
  ],
  collectCoverageFrom: [
    'components/**/*.{ts,tsx}',
    'lib/**/*.{ts,tsx}',
    'app/**/*.{ts,tsx}',
    'hooks/**/*.{ts,tsx}',
    '!**/*.d.ts',
    '!**/node_modules/**',
    '!**/.next/**',
    '!**/coverage/**',
  ],
  coverageThreshold: {
    global: {
      branches: 80,
      functions: 80,
      lines: 80,
      statements: 80,
    },
  },
  transform: {
    '^.+\\.(ts|tsx)$': ['ts-jest', {
      tsconfig: {
        jsx: 'react-jsx',
      },
    }],
  },
  testTimeout: 30000,
}

// createJestConfig is exported this way to ensure that next/jest can load the Next.js config which is async
export default createJestConfig(config)
\end{lstlisting}

\subsection{Utilitaires de Test}

\begin{lstlisting}[language=TypeScript, caption=__tests__/utils/test-utils.tsx]
import React, { ReactElement } from 'react'
import { render, RenderOptions } from '@testing-library/react'
import { ThemeProvider } from '@/components/theme-provider'

// Custom render function that includes providers
const AllTheProviders = ({ children }: { children: React.ReactNode }) => {
  return (
    <ThemeProvider
      attribute="class"
      defaultTheme="system"
      enableSystem
      disableTransitionOnChange
    >
      {children}
    </ThemeProvider>
  )
}

const customRender = (
  ui: ReactElement,
  options?: Omit<RenderOptions, 'wrapper'>
) => render(ui, { wrapper: AllTheProviders, ...options })

export * from '@testing-library/react'
export { customRender as render }

// Mock implementations for testing
export const mockUser = {
  id: 'test-user-1',
  name: 'John Doe',
  email: 'john.doe@example.com',
  profilePictureUrl: 'https://example.com/profile.jpg',
  createdAt: '2023-01-01T00:00:00.000Z',
  updatedAt: '2023-01-01T00:00:00.000Z',
}

export const mockUsers = [
  mockUser,
  {
    id: 'test-user-2',
    name: 'Jane Smith',
    email: 'jane.smith@example.com',
    profilePictureUrl: 'https://example.com/profile2.jpg',
    createdAt: '2023-01-02T00:00:00.000Z',
    updatedAt: '2023-01-02T00:00:00.000Z',
  },
]

// Mock file for testing file uploads
export const createMockFile = (
  name: string = 'test.jpg',
  type: string = 'image/jpeg',
  size: number = 1024
): File => {
  const file = new File(['test content'], name, { type })
  Object.defineProperty(file, 'size', { value: size })
  return file
}

// Mock fetch responses
export const mockFetchSuccess = (data: any) => {
  global.fetch = jest.fn().mockResolvedValue({
    ok: true,
    status: 200,
    json: jest.fn().mockResolvedValue(data),
  } as any)
}

export const mockFetchError = (status: number = 500, message: string = 'Internal Server Error') => {
  global.fetch = jest.fn().mockResolvedValue({
    ok: false,
    status,
    json: jest.fn().mockResolvedValue({ error: message }),
  } as any)
}

// Cleanup function
export const cleanupMocks = () => {
  jest.clearAllMocks()
  if (global.fetch) {
    (global.fetch as jest.Mock).mockRestore()
  }
}
\end{lstlisting}

\section{Scripts DevOps}

\subsection{Script de Test du Pipeline}

\begin{lstlisting}[language=bash, caption=scripts/test-pipeline.sh]
#!/bin/bash

set -e

echo "🧪 Testing CI/CD Pipeline Components"
echo "====================================="

# Colors
GREEN='\033[0;32m'
RED='\033[0;31m'
YELLOW='\033[1;33m'
NC='\033[0m'

# Test counters
PASSED=0
FAILED=0

# Function to run test and track results
run_test() {
    local test_name="$1"
    local test_command="$2"
    
    echo -e "${YELLOW}Testing: $test_name${NC}"
    
    if eval "$test_command"; then
        echo -e "${GREEN}✅ $test_name passed${NC}"
        ((PASSED++))
    else
        echo -e "${RED}❌ $test_name failed${NC}"
        ((FAILED++))
    fi
    echo ""
}

# Test Node.js and dependencies
run_test "Node.js version" "node --version"
run_test "pnpm installation" "pnpm --version"
run_test "Dependencies installation" "pnpm install --frozen-lockfile"

# Test code quality
run_test "TypeScript compilation" "pnpm run type-check"
run_test "ESLint checks" "pnpm run lint"

# Test application build
run_test "Next.js build" "pnpm run build"

# Test Docker build
run_test "Docker build" "docker build -t aws-next-express-test ."

# Test Kubernetes manifests
run_test "Kubernetes manifests validation" "kubectl apply --dry-run=client -f k8s/"

# Start DynamoDB for tests
echo -e "${YELLOW}Starting DynamoDB Local for tests...${NC}"
docker run -d --name dynamodb-test -p 8001:8000 amazon/dynamodb-local:latest > /dev/null 2>&1
sleep 3

# Test database operations
export DYNAMODB_ENDPOINT=http://localhost:8001
run_test "DynamoDB connection" "node scripts/init-dynamodb.js"

# Run tests
run_test "Unit tests" "pnpm test"
run_test "Test coverage" "pnpm run test:coverage"

# Cleanup
echo -e "${YELLOW}Cleaning up test resources...${NC}"
docker stop dynamodb-test > /dev/null 2>&1 || true
docker rm dynamodb-test > /dev/null 2>&1 || true
docker rmi aws-next-express-test > /dev/null 2>&1 || true

# Results
echo "====================================="
echo -e "${GREEN}Tests passed: $PASSED${NC}"
echo -e "${RED}Tests failed: $FAILED${NC}"

if [ $FAILED -eq 0 ]; then
    echo -e "${GREEN}🎉 All pipeline tests passed!${NC}"
    exit 0
else
    echo -e "${RED}💥 Some tests failed. Please check the output above.${NC}"
    exit 1
fi
\end{lstlisting}

\subsection{Script de Monitoring}

\begin{lstlisting}[language=bash, caption=scripts/monitor.sh]
#!/bin/bash

# Health check script for monitoring
check_health() {
    local service_name="$1"
    local health_url="$2"
    local expected_status="${3:-200}"
    
    echo "Checking $service_name health..."
    
    response=$(curl -s -w "%{http_code}" -o /dev/null "$health_url" || echo "000")
    
    if [ "$response" = "$expected_status" ]; then
        echo "✅ $service_name is healthy (HTTP $response)"
        return 0
    else
        echo "❌ $service_name is unhealthy (HTTP $response)"
        return 1
    fi
}

# Main monitoring loop
main() {
    local base_url="${1:-http://localhost:3000}"
    
    echo "🔍 AWS Next Express Health Monitor"
    echo "Monitoring: $base_url"
    echo "================================="
    
    # Check main application
    check_health "Application" "$base_url/api/health"
    
    # Check API endpoints
    check_health "Users API" "$base_url/api/users"
    
    # Check static assets
    check_health "Static Assets" "$base_url/_next/static/css"
    
    echo "Monitoring check completed at $(date)"
}

# Run monitoring
main "$@"
\end{lstlisting}

\section{Configuration de Sécurité}

\subsection{Security Headers Middleware}

\begin{lstlisting}[language=TypeScript, caption=middleware.ts]
import { NextRequest, NextResponse } from 'next/server'

export function middleware(request: NextRequest) {
  // Security headers
  const response = NextResponse.next()
  
  // Prevent clickjacking
  response.headers.set('X-Frame-Options', 'DENY')
  
  // Prevent MIME type sniffing
  response.headers.set('X-Content-Type-Options', 'nosniff')
  
  // Enable XSS protection
  response.headers.set('X-XSS-Protection', '1; mode=block')
  
  // Force HTTPS
  response.headers.set(
    'Strict-Transport-Security',
    'max-age=31536000; includeSubDomains; preload'
  )
  
  // Content Security Policy
  response.headers.set(
    'Content-Security-Policy',
    [
      "default-src 'self'",
      "script-src 'self' 'unsafe-inline' 'unsafe-eval'",
      "style-src 'self' 'unsafe-inline'",
      "img-src 'self' data: https:",
      "font-src 'self'",
      "connect-src 'self' https://api.github.com",
      "frame-ancestors 'none'",
    ].join('; ')
  )
  
  // Referrer Policy
  response.headers.set('Referrer-Policy', 'strict-origin-when-cross-origin')
  
  // Permissions Policy
  response.headers.set(
    'Permissions-Policy',
    'geolocation=(), microphone=(), camera=()'
  )
  
  return response
}

export const config = {
  matcher: [
    /*
     * Match all request paths except for the ones starting with:
     * - api (API routes)
     * - _next/static (static files)
     * - _next/image (image optimization files)
     * - favicon.ico (favicon file)
     */
    '/((?!api|_next/static|_next/image|favicon.ico).*)',
  ],
}
\end{lstlisting} 