\chapter{Bibliographie et Références}

\section{Références Techniques}

\subsection{Documentation Officielle}

\begin{enumerate}
    \item \textbf{Next.js Documentation} \\
    \textit{The React Framework for Production} \\
    Vercel Inc. \\
    \url{https://nextjs.org/docs} \\
    Consulté en novembre 2024

    \item \textbf{AWS DynamoDB Developer Guide} \\
    \textit{Amazon DynamoDB Documentation} \\
    Amazon Web Services \\
    \url{https://docs.aws.amazon.com/dynamodb/} \\
    Consulté en novembre 2024

    \item \textbf{AWS SDK for JavaScript v3} \\
    \textit{Developer Guide} \\
    Amazon Web Services \\
    \url{https://docs.aws.amazon.com/AWSJavaScriptSDK/v3/latest/} \\
    Consulté en novembre 2024

    \item \textbf{Docker Documentation} \\
    \textit{Get Started with Docker} \\
    Docker Inc. \\
    \url{https://docs.docker.com/} \\
    Consulté en novembre 2024

    \item \textbf{Kubernetes Documentation} \\
    \textit{Production-Grade Container Orchestration} \\
    Cloud Native Computing Foundation \\
    \url{https://kubernetes.io/docs/} \\
    Consulté en novembre 2024

    \item \textbf{ArgoCD Documentation} \\
    \textit{Declarative GitOps CD for Kubernetes} \\
    Argo Project \\
    \url{https://argo-cd.readthedocs.io/} \\
    Consulté en novembre 2024

    \item \textbf{GitHub Actions Documentation} \\
    \textit{Automate your workflow from idea to production} \\
    GitHub Inc. \\
    \url{https://docs.github.com/en/actions} \\
    Consulté en novembre 2024

    \item \textbf{TypeScript Documentation} \\
    \textit{JavaScript that scales} \\
    Microsoft Corporation \\
    \url{https://www.typescriptlang.org/docs/} \\
    Consulté en novembre 2024

    \item \textbf{Tailwind CSS Documentation} \\
    \textit{A utility-first CSS framework} \\
    Tailwind Labs Inc. \\
    \url{https://tailwindcss.com/docs} \\
    Consulté en novembre 2024

    \item \textbf{Jest Documentation} \\
    \textit{Delightful JavaScript Testing} \\
    Meta Platforms, Inc. \\
    \url{https://jestjs.io/docs/getting-started} \\
    Consulté en novembre 2024
\end{enumerate}

\subsection{Références Méthodologiques}

\begin{enumerate}
    \item \textbf{The Scrum Guide} \\
    Ken Schwaber et Jeff Sutherland \\
    \textit{The Definitive Guide to Scrum: The Rules of the Game} \\
    Scrum.org, 2020 \\
    \url{https://scrumguides.org/}

    \item \textbf{Agile Manifesto} \\
    Beck, K., Beedle, M., van Bennekum, A., et al. \\
    \textit{Manifesto for Agile Software Development} \\
    2001 \\
    \url{https://agilemanifesto.org/}

    \item \textbf{DevOps Handbook} \\
    Gene Kim, Jez Humble, Patrick Debois, John Willis \\
    \textit{The DevOps Handbook: How to Create World-Class Agility, Reliability, and Security} \\
    IT Revolution Press, 2016

    \item \textbf{Continuous Delivery} \\
    Jez Humble et David Farley \\
    \textit{Continuous Delivery: Reliable Software Releases through Build, Test, and Deployment Automation} \\
    Addison-Wesley Professional, 2010

    \item \textbf{Building Microservices} \\
    Sam Newman \\
    \textit{Building Microservices: Designing Fine-Grained Systems} \\
    O'Reilly Media, 2021
\end{enumerate}

\subsection{Articles et Publications}

\begin{enumerate}
    \item \textbf{Twelve-Factor App} \\
    Adam Wiggins \\
    \textit{The twelve-factor app methodology} \\
    Heroku, 2011 \\
    \url{https://12factor.net/}

    \item \textbf{The Phoenix Project} \\
    Gene Kim, Kevin Behr, George Spafford \\
    \textit{A Novel about IT, DevOps, and Helping Your Business Win} \\
    IT Revolution Press, 2018

    \item \textbf{Cloud Native Computing Foundation} \\
    \textit{Cloud Native Definition} \\
    CNCF \\
    \url{https://github.com/cncf/toc/blob/main/DEFINITION.md}

    \item \textbf{Martin Fowler's Blog} \\
    Martin Fowler \\
    \textit{Microservices, Continuous Integration, DevOps} \\
    ThoughtWorks \\
    \url{https://martinfowler.com/}

    \item \textbf{The State of DevOps Report} \\
    DORA (DevOps Research and Assessment) \\
    \textit{Accelerate State of DevOps 2023} \\
    Google Cloud, 2023
\end{enumerate}

\section{Outils et Technologies}

\subsection{Frameworks et Bibliothèques}

\begin{table}[H]
    \centering
    \begin{tabularx}{\textwidth}{|l|X|l|}
        \hline
        \textbf{Technologie} & \textbf{Description} & \textbf{Version} \\
        \hline
        Next.js & Framework React full-stack & 15.2.4 \\
        \hline
        React & Bibliothèque JavaScript pour interfaces utilisateur & 18.3.1 \\
        \hline
        TypeScript & JavaScript avec typage statique & 5.8.3 \\
        \hline
        Tailwind CSS & Framework CSS utility-first & 3.4.17 \\
        \hline
        Zod & Validation de schémas TypeScript & 3.23.8 \\
        \hline
        AWS SDK v3 & Kit de développement AWS pour JavaScript & 3.621.0 \\
        \hline
        Radix UI & Composants UI accessibles & 1.1.1 \\
        \hline
        Lucide React & Icônes React & 0.445.0 \\
        \hline
    \end{tabularx}
    \caption{Frameworks et bibliothèques utilisés}
    \label{tab:frameworks}
\end{table}

\subsection{Outils de Développement}

\begin{table}[H]
    \centering
    \begin{tabularx}{\textwidth}{|l|X|l|}
        \hline
        \textbf{Outil} & \textbf{Description} & \textbf{Version} \\
        \hline
        Node.js & Runtime JavaScript & 18.17.0 \\
        \hline
        pnpm & Gestionnaire de paquets & 8.6.12 \\
        \hline
        ESLint & Linter JavaScript/TypeScript & 8.57.1 \\
        \hline
        Prettier & Formateur de code & 3.0.0 \\
        \hline
        Jest & Framework de tests & 29.7.0 \\
        \hline
        Playwright & Tests end-to-end & 1.47.2 \\
        \hline
        Husky & Git hooks & 8.0.0 \\
        \hline
        Docker & Containerisation & 24.0.5 \\
        \hline
    \end{tabularx}
    \caption{Outils de développement utilisés}
    \label{tab:dev_tools_ref}
\end{table}

\subsection{Infrastructure et DevOps}

\begin{table}[H]
    \centering
    \begin{tabularx}{\textwidth}{|l|X|l|}
        \hline
        \textbf{Service/Outil} & \textbf{Description} & \textbf{Version} \\
        \hline
        AWS DynamoDB & Base de données NoSQL managée & Latest \\
        \hline
        AWS S3 & Stockage d'objets & Latest \\
        \hline
        Kubernetes & Orchestration de conteneurs & 1.28+ \\
        \hline
        ArgoCD & GitOps pour Kubernetes & 2.8+ \\
        \hline
        GitHub Actions & CI/CD & Latest \\
        \hline
        Prometheus & Monitoring et métriques & Latest \\
        \hline
        Grafana & Visualisation de données & Latest \\
        \hline
        Terraform & Infrastructure as Code & 1.5+ \\
        \hline
    \end{tabularx}
    \caption{Infrastructure et outils DevOps}
    \label{tab:infrastructure_ref}
\end{table}

\section{Standards et Bonnes Pratiques}

\subsection{Standards Web}

\begin{enumerate}
    \item \textbf{Web Content Accessibility Guidelines (WCAG) 2.1} \\
    W3C \\
    \textit{Guidelines for making web content accessible} \\
    \url{https://www.w3.org/WAI/WCAG21/quickref/}

    \item \textbf{Core Web Vitals} \\
    Google \\
    \textit{Essential metrics for a healthy site} \\
    \url{https://web.dev/vitals/}

    \item \textbf{Progressive Web App (PWA)} \\
    Google Developers \\
    \textit{Progressive Web Apps guidelines} \\
    \url{https://web.dev/progressive-web-apps/}

    \item \textbf{OWASP Top 10} \\
    OWASP Foundation \\
    \textit{Top 10 Web Application Security Risks} \\
    \url{https://owasp.org/www-project-top-ten/}
\end{enumerate}

\subsection{Standards DevOps}

\begin{enumerate}
    \item \textbf{DORA Metrics} \\
    DevOps Research and Assessment \\
    \textit{Four Key Metrics for DevOps Performance} \\
    - Deployment Frequency \\
    - Lead Time for Changes \\
    - Change Failure Rate \\
    - Time to Recovery

    \item \textbf{GitOps Principles} \\
    Weaveworks \\
    \textit{GitOps: Operations by Pull Request} \\
    \url{https://www.gitops.tech/}

    \item \textbf{Container Image Security} \\
    NIST \\
    \textit{Application Container Security Guide} \\
    NIST Special Publication 800-190

    \item \textbf{Kubernetes Security Best Practices} \\
    CNCF \\
    \textit{Kubernetes Security Best Practices} \\
    \url{https://kubernetes.io/docs/concepts/security/}
\end{enumerate}

\section{Ressources Pédagogiques}

\subsection{Cours et Formations}

\begin{enumerate}
    \item \textbf{AWS Training and Certification} \\
    Amazon Web Services \\
    \textit{Cloud computing courses and certifications} \\
    \url{https://aws.amazon.com/training/}

    \item \textbf{Kubernetes Documentation Tutorials} \\
    CNCF \\
    \textit{Learning Kubernetes Basics} \\
    \url{https://kubernetes.io/docs/tutorials/}

    \item \textbf{React Official Tutorial} \\
    Meta \\
    \textit{Tutorial: Intro to React} \\
    \url{https://reactjs.org/tutorial/tutorial.html}

    \item \textbf{TypeScript Handbook} \\
    Microsoft \\
    \textit{The TypeScript Handbook} \\
    \url{https://www.typescriptlang.org/docs/}

    \item \textbf{Docker Get Started} \\
    Docker Inc. \\
    \textit{Get started with Docker} \\
    \url{https://docs.docker.com/get-started/}
\end{enumerate}

\subsection{Communautés et Forums}

\begin{enumerate}
    \item \textbf{Stack Overflow} \\
    \textit{Programming Q\&A platform} \\
    \url{https://stackoverflow.com/}

    \item \textbf{GitHub Discussions} \\
    \textit{AWS Next Express repository discussions} \\
    \url{https://github.com/nourhb/aws-next-express}

    \item \textbf{Reddit - r/nextjs} \\
    \textit{Next.js community discussions} \\
    \url{https://reddit.com/r/nextjs}

    \item \textbf{Discord - Reactiflux} \\
    \textit{React developers community} \\
    \url{https://discord.gg/reactiflux}

    \item \textbf{CNCF Slack} \\
    \textit{Cloud Native Computing Foundation community} \\
    \url{https://slack.cncf.io/}
\end{enumerate}

\section{Remerciements}

Nous tenons à remercier particulièrement :

\begin{itemize}
    \item L'\textbf{équipe pédagogique d'ITEAM University} pour l'encadrement et les conseils techniques
    \item La \textbf{communauté open source} pour les outils et frameworks utilisés
    \item Les \textbf{mainteneurs des projets} Next.js, React, Kubernetes, et AWS SDK
    \item Les \textbf{contributeurs de la documentation} pour les guides et tutoriels
    \item La \textbf{communauté DevOps} pour le partage de bonnes pratiques
\end{itemize}

\section{Acronymes et Abréviations}

\begin{table}[H]
    \centering
    \begin{tabularx}{\textwidth}{|l|X|}
        \hline
        \textbf{Acronyme} & \textbf{Signification} \\
        \hline
        API & Application Programming Interface \\
        \hline
        AWS & Amazon Web Services \\
        \hline
        CI/CD & Continuous Integration / Continuous Deployment \\
        \hline
        CRUD & Create, Read, Update, Delete \\
        \hline
        CSS & Cascading Style Sheets \\
        \hline
        DOM & Document Object Model \\
        \hline
        E2E & End-to-End \\
        \hline
        HTML & HyperText Markup Language \\
        \hline
        HTTP & HyperText Transfer Protocol \\
        \hline
        IDE & Integrated Development Environment \\
        \hline
        JSON & JavaScript Object Notation \\
        \hline
        JWT & JSON Web Token \\
        \hline
        MVP & Minimum Viable Product \\
        \hline
        NoSQL & Not Only SQL \\
        \hline
        REST & Representational State Transfer \\
        \hline
        S3 & Simple Storage Service \\
        \hline
        SPA & Single Page Application \\
        \hline
        SQL & Structured Query Language \\
        \hline
        SSR & Server-Side Rendering \\
        \hline
        TLS & Transport Layer Security \\
        \hline
        UI/UX & User Interface / User Experience \\
        \hline
        URL & Uniform Resource Locator \\
        \hline
        YAML & YAML Ain't Markup Language \\
        \hline
    \end{tabularx}
    \caption{Acronymes et abréviations utilisés}
    \label{tab:acronyms}
\end{table}

Ce projet AWS Next Express s'appuie sur un écosystème riche de technologies, d'outils et de méthodologies éprouvées, démontrant l'importance de la collaboration communautaire dans le développement logiciel moderne. 