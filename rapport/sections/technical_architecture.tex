\chapter{Architecture Technique et Conception}

\section{Vue d'Ensemble de l'Architecture}

\subsection{Architecture Globale}

L'architecture d'AWS Next Express suit le pattern d'architecture moderne en couches, optimisée pour le cloud et la scalabilité. Le système est conçu selon une approche microservices avec séparation claire des responsabilités.

\begin{figure}[H]
    \centering
    \includegraphics[width=1.0\textwidth]{images/system_architecture.png}
    \caption{Architecture globale du système AWS Next Express}
    \label{fig:system_architecture}
\end{figure}

\subsection{Principes Architecturaux}

Notre architecture respecte les principes suivants :

\begin{enumerate}
    \item \textbf{Séparation des préoccupations} : Chaque couche a une responsabilité spécifique
    \item \textbf{Scalabilité horizontale} : Possibilité d'ajouter des instances selon la charge
    \item \textbf{Résilience} : Tolérance aux pannes avec récupération automatique
    \item \textbf{Sécurité by design} : Sécurité intégrée à tous les niveaux
    \item \textbf{Observabilité} : Monitoring et logging pour la supervision
\end{enumerate}

\section{Architecture Frontend}

\subsection{Architecture React/Next.js}

\subsubsection{Structure des Composants}

L'architecture frontend suit le pattern de composition avec une hiérarchie claire :

\begin{lstlisting}[language=bash, caption=Structure des composants]
components/
├── ui/                    # Composants de base (shadcn/ui)
│   ├── button.tsx
│   ├── card.tsx
│   ├── input.tsx
│   └── ...
├── layout/               # Composants de mise en page
│   ├── navbar.tsx
│   ├── sidebar.tsx
│   └── footer.tsx
├── features/            # Composants métier
│   ├── user-list.tsx
│   ├── user-form.tsx
│   └── dashboard.tsx
└── common/             # Composants communs
    ├── loading.tsx
    ├── error-boundary.tsx
    └── theme-provider.tsx
\end{lstlisting}

\subsubsection{Pattern de Composition}

\begin{lstlisting}[language=TypeScript, caption=Exemple de composition de composants]
// Composant container
function UserManagementPage() {
  return (
    <Layout>
      <Header title="Gestion des Utilisateurs" />
      <Breadcrumb items={breadcrumbItems} />
      <UserList />
      <FloatingActionButton href="/users/new" />
    </Layout>
  );
}

// Composant de présentation
function UserList({ users, onDelete, onEdit }) {
  return (
    <Grid>
      {users.map(user => (
        <UserCard 
          key={user.id}
          user={user}
          onDelete={() => onDelete(user.id)}
          onEdit={() => onEdit(user.id)}
        />
      ))}
    </Grid>
  );
}
\end{lstlisting}

\subsection{Gestion d'État}

\subsubsection{État Local vs État Global}

\begin{table}[H]
    \centering
    \begin{tabularx}{\textwidth}{|X|X|X|}
        \hline
        \textbf{Type d'État} & \textbf{Mécanisme} & \textbf{Usage} \\
        \hline
        État de composant & useState, useReducer & Formulaires, UI temporaire \\
        \hline
        État partagé & Context API & Thème, authentification \\
        \hline
        État serveur & SWR, React Query & Données backend, cache \\
        \hline
        État de navigation & Next.js Router & Routing, navigation \\
        \hline
    \end{tabularx}
    \caption{Stratégies de gestion d'état}
    \label{tab:state_management}
\end{table}

\subsubsection{Pattern de Hooks Personnalisés}

\begin{lstlisting}[language=TypeScript, caption=Hook personnalisé pour la gestion des utilisateurs]
function useUsers() {
  const [users, setUsers] = useState<User[]>([]);
  const [loading, setLoading] = useState(true);
  const [error, setError] = useState<string | null>(null);

  const fetchUsers = useCallback(async () => {
    try {
      setLoading(true);
      const response = await api.get('/users');
      setUsers(response.data);
    } catch (err) {
      setError(err.message);
    } finally {
      setLoading(false);
    }
  }, []);

  const createUser = useCallback(async (userData) => {
    const response = await api.post('/users', userData);
    setUsers(prev => [...prev, response.data]);
    return response.data;
  }, []);

  return { users, loading, error, fetchUsers, createUser };
}
\end{lstlisting}

\section{Architecture Backend}

\subsection{API Routes Architecture}

\subsubsection{Structure des API Routes}

\begin{lstlisting}[language=bash, caption=Organisation des API Routes]
app/api/
├── users/
│   ├── route.ts           # GET /api/users, POST /api/users
│   └── [id]/
│       └── route.ts       # GET, PUT, DELETE /api/users/[id]
├── files/
│   ├── route.ts           # File management endpoints
│   └── [id]/
│       └── route.ts
└── health/
    └── route.ts           # Health check endpoint
\end{lstlisting}

\subsubsection{Pattern de Contrôleur}

\begin{lstlisting}[language=TypeScript, caption=Structure d'un contrôleur API]
// app/api/users/route.ts
import { NextRequest, NextResponse } from 'next/server';
import { UserService } from '@/lib/services/user-service';
import { validateRequest } from '@/lib/middleware/validation';
import { handleError } from '@/lib/middleware/error-handler';

export async function GET(request: NextRequest) {
  try {
    const users = await UserService.getAllUsers();
    return NextResponse.json(users);
  } catch (error) {
    return handleError(error);
  }
}

export async function POST(request: NextRequest) {
  try {
    const body = await request.json();
    const validatedData = await validateRequest(body, userSchema);
    const user = await UserService.createUser(validatedData);
    return NextResponse.json(user, { status: 201 });
  } catch (error) {
    return handleError(error);
  }
}
\end{lstlisting}

\subsection{Couche de Services}

\subsubsection{Séparation Logique}

\begin{lstlisting}[language=TypeScript, caption=Architecture de service]
// lib/services/user-service.ts
export class UserService {
  private userRepository: UserRepository;
  private s3Service: S3Service;
  private validationService: ValidationService;

  constructor() {
    this.userRepository = new UserRepository();
    this.s3Service = new S3Service();
    this.validationService = new ValidationService();
  }

  async createUser(userData: CreateUserRequest): Promise<User> {
    // 1. Validation
    await this.validationService.validateUser(userData);
    
    // 2. Upload de l'image
    const imageUrl = await this.s3Service.uploadImage(userData.image);
    
    // 3. Sauvegarde en base
    const user = await this.userRepository.create({
      ...userData,
      profilePictureUrl: imageUrl,
    });
    
    // 4. Événements/notifications
    await this.notificationService.notifyUserCreated(user);
    
    return user;
  }
}
\end{lstlisting}

\section{Architecture de Données}

\subsection{Modélisation DynamoDB}

\subsubsection{Conception des Tables}

\begin{table}[H]
    \centering
    \begin{tabularx}{\textwidth}{|l|X|X|X|}
        \hline
        \textbf{Table} & \textbf{Partition Key} & \textbf{Sort Key} & \textbf{GSI} \\
        \hline
        Users & id (String) & - & email-index (email) \\
        \hline
        Files & userId (String) & fileId (String) & type-index (fileType) \\
        \hline
        Sessions & sessionId (String) & - & user-index (userId) \\
        \hline
    \end{tabularx}
    \caption{Structure des tables DynamoDB}
    \label{tab:dynamodb_tables}
\end{table}

\subsubsection{Patterns d'Accès aux Données}

\begin{lstlisting}[language=TypeScript, caption=Patterns DynamoDB]
// Pattern 1: Accès par clé primaire
async getUserById(id: string): Promise<User | null> {
  const result = await this.dynamodb.getItem({
    TableName: 'Users',
    Key: { id }
  });
  return result.Item as User || null;
}

// Pattern 2: Requête avec GSI
async getUserByEmail(email: string): Promise<User | null> {
  const result = await this.dynamodb.query({
    TableName: 'Users',
    IndexName: 'email-index',
    KeyConditionExpression: 'email = :email',
    ExpressionAttributeValues: { ':email': email }
  });
  return result.Items?.[0] as User || null;
}

// Pattern 3: Scan avec filtres
async searchUsers(criteria: SearchCriteria): Promise<User[]> {
  const result = await this.dynamodb.scan({
    TableName: 'Users',
    FilterExpression: 'contains(#name, :searchTerm)',
    ExpressionAttributeNames: { '#name': 'name' },
    ExpressionAttributeValues: { ':searchTerm': criteria.name }
  });
  return result.Items as User[];
}
\end{lstlisting}

\subsection{Gestion des Fichiers avec S3}

\subsubsection{Architecture de Stockage}

\begin{lstlisting}[language=bash, caption=Organisation des buckets S3]
aws-next-express-storage/
├── profile-pictures/
│   ├── original/
│   │   └── {userId}/{timestamp}-{filename}
│   ├── thumbnails/
│   │   └── {userId}/{timestamp}-thumb-{filename}
│   └── compressed/
│       └── {userId}/{timestamp}-comp-{filename}
├── documents/
│   └── {userId}/{category}/{filename}
└── temp/
    └── {uploadId}/{filename}
\end{lstlisting}

\subsubsection{Service de Gestion de Fichiers}

\begin{lstlisting}[language=TypeScript, caption=Service S3]
export class S3Service {
  private s3Client: S3Client;
  private bucketName: string;

  async uploadFile(file: File, options: UploadOptions): Promise<string> {
    const key = this.generateFileKey(file, options);
    
    // Upload principal
    await this.s3Client.send(new PutObjectCommand({
      Bucket: this.bucketName,
      Key: key,
      Body: file.buffer,
      ContentType: file.mimetype,
      Metadata: {
        originalName: file.originalname,
        uploadedBy: options.userId,
        uploadedAt: new Date().toISOString(),
      }
    }));

    // Génération de thumbnail si image
    if (this.isImage(file)) {
      await this.generateThumbnail(key, file);
    }

    return this.getPublicUrl(key);
  }

  private generateFileKey(file: File, options: UploadOptions): string {
    const timestamp = Date.now();
    const extension = path.extname(file.originalname);
    const safeName = this.sanitizeFilename(file.originalname);
    
    return `${options.folder}/${options.userId}/${timestamp}-${safeName}${extension}`;
  }
}
\end{lstlisting}

\section{Architecture de Sécurité}

\subsection{Stratégie de Sécurité}

\subsubsection{Couches de Sécurité}

\begin{enumerate}
    \item \textbf{Frontend} : Validation côté client, sanitisation des inputs
    \item \textbf{API Gateway} : Rate limiting, authentification JWT
    \item \textbf{Application} : Validation serveur, autorisation RBAC
    \item \textbf{Infrastructure} : VPC, Security Groups, IAM roles
    \item \textbf{Données} : Chiffrement au repos et en transit
\end{enumerate}

\subsubsection{Implémentation de la Sécurité}

\begin{lstlisting}[language=TypeScript, caption=Middleware de sécurité]
// lib/middleware/security.ts
export function securityMiddleware(request: NextRequest) {
  // CORS
  const corsHeaders = {
    'Access-Control-Allow-Origin': process.env.ALLOWED_ORIGINS,
    'Access-Control-Allow-Methods': 'GET,POST,PUT,DELETE,OPTIONS',
    'Access-Control-Allow-Headers': 'Content-Type,Authorization',
  };

  // Security Headers
  const securityHeaders = {
    'X-Content-Type-Options': 'nosniff',
    'X-Frame-Options': 'DENY',
    'X-XSS-Protection': '1; mode=block',
    'Strict-Transport-Security': 'max-age=31536000; includeSubDomains',
    'Content-Security-Policy': "default-src 'self'; script-src 'self' 'unsafe-inline'",
  };

  // Rate Limiting
  const clientIP = request.ip;
  if (await isRateLimited(clientIP)) {
    return new Response('Rate limit exceeded', { status: 429 });
  }

  return { corsHeaders, securityHeaders };
}
\end{lstlisting}

\subsection{Authentification et Autorisation}

\subsubsection{Flux d'Authentification}

\begin{figure}[H]
    \centering
    \includegraphics[width=0.9\textwidth]{images/auth_flow.png}
    \caption{Flux d'authentification et autorisation}
    \label{fig:auth_flow}
\end{figure}

\subsubsection{Gestion des Tokens JWT}

\begin{lstlisting}[language=TypeScript, caption=Service d'authentification]
export class AuthService {
  private jwtSecret: string;
  private tokenExpiry: string;

  async generateToken(user: User): Promise<string> {
    const payload = {
      userId: user.id,
      email: user.email,
      role: user.role,
      iat: Math.floor(Date.now() / 1000),
      exp: Math.floor(Date.now() / 1000) + (24 * 60 * 60), // 24h
    };

    return jwt.sign(payload, this.jwtSecret, {
      algorithm: 'HS256',
      issuer: 'aws-next-express',
      audience: 'aws-next-express-users',
    });
  }

  async verifyToken(token: string): Promise<TokenPayload | null> {
    try {
      const decoded = jwt.verify(token, this.jwtSecret) as TokenPayload;
      
      // Vérification de la validité
      if (decoded.exp < Math.floor(Date.now() / 1000)) {
        throw new Error('Token expired');
      }

      return decoded;
    } catch (error) {
      console.error('Token verification failed:', error);
      return null;
    }
  }
}
\end{lstlisting}

\section{Architecture de Performance}

\subsection{Stratégies d'Optimisation}

\subsubsection{Frontend Performance}

\begin{table}[H]
    \centering
    \begin{tabularx}{\textwidth}{|X|X|X|}
        \hline
        \textbf{Technique} & \textbf{Implémentation} & \textbf{Bénéfice} \\
        \hline
        Code Splitting & Dynamic imports, Route-based splitting & Réduction du bundle initial \\
        \hline
        Image Optimization & Next.js Image component & Lazy loading, formats optimisés \\
        \hline
        Static Generation & SSG pour pages statiques & Temps de chargement ultra-rapide \\
        \hline
        Server-Side Rendering & SSR pour contenu dynamique & SEO amélioré, première peinture rapide \\
        \hline
        Caching & SWR, React Query & Réduction des appels API \\
        \hline
    \end{tabularx}
    \caption{Techniques d'optimisation frontend}
    \label{tab:frontend_optimization}
\end{table}

\subsubsection{Backend Performance}

\begin{lstlisting}[language=TypeScript, caption=Caching strategy]
// lib/cache/redis-cache.ts
export class CacheService {
  private redis: RedisClient;

  async get<T>(key: string): Promise<T | null> {
    const cached = await this.redis.get(key);
    return cached ? JSON.parse(cached) : null;
  }

  async set<T>(key: string, value: T, ttl: number = 3600): Promise<void> {
    await this.redis.setex(key, ttl, JSON.stringify(value));
  }

  async invalidatePattern(pattern: string): Promise<void> {
    const keys = await this.redis.keys(pattern);
    if (keys.length > 0) {
      await this.redis.del(...keys);
    }
  }
}

// Usage avec décorateur
@Cacheable('users:*', 300) // Cache 5 minutes
async getUserById(id: string): Promise<User> {
  return await this.userRepository.findById(id);
}
\end{lstlisting}

\subsection{Monitoring et Observabilité}

\subsubsection{Métriques Applicatives}

\begin{lstlisting}[language=TypeScript, caption=Service de métriques]
export class MetricsService {
  private prometheus: PrometheusRegistry;

  constructor() {
    this.httpRequestDuration = new Histogram({
      name: 'http_request_duration_seconds',
      help: 'Duration of HTTP requests in seconds',
      labelNames: ['method', 'route', 'status_code'],
      buckets: [0.1, 0.5, 1, 2, 5],
    });

    this.activeUsers = new Gauge({
      name: 'active_users_total',
      help: 'Number of active users',
    });

    this.databaseConnections = new Gauge({
      name: 'database_connections_active',
      help: 'Number of active database connections',
    });
  }

  recordHttpRequest(method: string, route: string, statusCode: number, duration: number) {
    this.httpRequestDuration
      .labels(method, route, statusCode.toString())
      .observe(duration);
  }

  updateActiveUsers(count: number) {
    this.activeUsers.set(count);
  }
}
\end{lstlisting}

\section{Architecture d'Évolutivité}

\subsection{Patterns de Scalabilité}

\subsubsection{Horizontal Scaling}

\begin{figure}[H]
    \centering
    \includegraphics[width=0.9\textwidth]{images/scaling_architecture.png}
    \caption{Architecture de mise à l'échelle horizontale}
    \label{fig:scaling_architecture}
\end{figure}

\subsubsection{Database Sharding Strategy}

\begin{lstlisting}[language=TypeScript, caption=Stratégie de sharding]
export class ShardingService {
  private shards: Map<string, DynamoDBClient>;

  constructor() {
    this.shards = new Map([
      ['shard-1', new DynamoDBClient({ region: 'us-east-1' })],
      ['shard-2', new DynamoDBClient({ region: 'us-west-2' })],
      ['shard-3', new DynamoDBClient({ region: 'eu-west-1' })],
    ]);
  }

  getShardForUser(userId: string): string {
    const hash = this.hashFunction(userId);
    const shardNumber = hash % this.shards.size;
    return `shard-${shardNumber + 1}`;
  }

  async getUserFromShard(userId: string): Promise<User> {
    const shardId = this.getShardForUser(userId);
    const client = this.shards.get(shardId);
    
    return await client.send(new GetItemCommand({
      TableName: `users-${shardId}`,
      Key: { id: { S: userId } }
    }));
  }

  private hashFunction(input: string): number {
    // Implémentation d'une fonction de hachage cohérente
    let hash = 0;
    for (let i = 0; i < input.length; i++) {
      const char = input.charCodeAt(i);
      hash = ((hash << 5) - hash) + char;
      hash = hash & hash; // Convert to 32-bit integer
    }
    return Math.abs(hash);
  }
}
\end{lstlisting}

Cette architecture technique robuste et évolutive constitue la fondation solide de notre application AWS Next Express, permettant de répondre aux exigences de performance, sécurité et scalabilité d'une application moderne. 