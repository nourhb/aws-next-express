\chapter{Implémentation et Développement}

\section{Environnement de Développement}

\subsection{Configuration de l'Environnement}

L'environnement de développement a été configuré pour maximiser la productivité et garantir la cohérence entre les développeurs.

\subsubsection{Stack de Développement}

\begin{table}[H]
    \centering
    \begin{tabularx}{\textwidth}{|l|X|X|}
        \hline
        \textbf{Outil} & \textbf{Version} & \textbf{Usage} \\
        \hline
        Node.js & 18.17.0 & Runtime JavaScript \\
        \hline
        pnpm & 8.6.12 & Gestionnaire de paquets \\
        \hline
        TypeScript & 5.8.3 & Langage de développement \\
        \hline
        Next.js & 15.2.4 & Framework React \\
        \hline
        Tailwind CSS & 3.4.17 & Framework CSS \\
        \hline
        ESLint & 8.57.1 & Linting JavaScript/TypeScript \\
        \hline
        Prettier & 3.0.0 & Formatage de code \\
        \hline
        Jest & 29.7.0 & Framework de tests \\
        \hline
        Docker & 24.0.5 & Containerisation \\
        \hline
    \end{tabularx}
    \caption{Outils de développement utilisés}
    \label{tab:dev_tools}
\end{table}

\subsubsection{Configuration IDE}

\begin{lstlisting}[language=JSON, caption=Configuration VSCode (.vscode/settings.json)]
{
  "typescript.preferences.importModuleSpecifier": "relative",
  "editor.formatOnSave": true,
  "editor.defaultFormatter": "esbenp.prettier-vscode",
  "editor.codeActionsOnSave": {
    "source.fixAll.eslint": true,
    "source.organizeImports": true
  },
  "files.associations": {
    "*.css": "tailwindcss"
  },
  "tailwindCSS.includeLanguages": {
    "typescript": "javascript",
    "typescriptreact": "javascript"
  },
  "emmet.includeLanguages": {
    "javascript": "javascriptreact",
    "typescript": "typescriptreact"
  }
}
\end{lstlisting}

\section{Développement Frontend}

\subsection{Configuration Next.js}

\subsubsection{Configuration du Projet}

\begin{lstlisting}[language=TypeScript, caption=Configuration Next.js (next.config.mjs)]
/** @type {import('next').NextConfig} */
const nextConfig = {
  typescript: {
    ignoreBuildErrors: false,
  },
  eslint: {
    ignoreDuringBuilds: false,
  },
  experimental: {
    appDir: true,
    serverComponentsExternalPackages: ['@aws-sdk'],
  },
  images: {
    domains: ['aws-next-express-bucket.s3.amazonaws.com'],
    formats: ['image/webp', 'image/avif'],
  },
  env: {
    CUSTOM_KEY: process.env.CUSTOM_KEY,
  },
};

export default nextConfig;
\end{lstlisting}

\subsubsection{Configuration TypeScript}

\begin{lstlisting}[language=JSON, caption=Configuration TypeScript (tsconfig.json)]
{
  "compilerOptions": {
    "target": "es5",
    "lib": ["dom", "dom.iterable", "es6"],
    "allowJs": true,
    "skipLibCheck": true,
    "strict": true,
    "noEmit": true,
    "esModuleInterop": true,
    "module": "esnext",
    "moduleResolution": "bundler",
    "resolveJsonModule": true,
    "isolatedModules": true,
    "jsx": "preserve",
    "incremental": true,
    "plugins": [
      {
        "name": "next"
      }
    ],
    "baseUrl": ".",
    "paths": {
      "@/*": ["./*"]
    }
  },
  "include": ["next-env.d.ts", "**/*.ts", "**/*.tsx", ".next/types/**/*.ts"],
  "exclude": ["node_modules"]
}
\end{lstlisting}

\subsection{Système de Design}

\subsubsection{Configuration Tailwind CSS}

\begin{lstlisting}[language=TypeScript, caption=Configuration Tailwind (tailwind.config.ts)]
import type { Config } from "tailwindcss"

const config = {
  darkMode: ["class"],
  content: [
    './pages/**/*.{ts,tsx}',
    './components/**/*.{ts,tsx}',
    './app/**/*.{ts,tsx}',
    './src/**/*.{ts,tsx}',
  ],
  prefix: "",
  theme: {
    container: {
      center: true,
      padding: "2rem",
      screens: {
        "2xl": "1400px",
      },
    },
    extend: {
      colors: {
        border: "hsl(var(--border))",
        input: "hsl(var(--input))",
        ring: "hsl(var(--ring))",
        background: "hsl(var(--background))",
        foreground: "hsl(var(--foreground))",
        primary: {
          DEFAULT: "hsl(var(--primary))",
          foreground: "hsl(var(--primary-foreground))",
        },
        secondary: {
          DEFAULT: "hsl(var(--secondary))",
          foreground: "hsl(var(--secondary-foreground))",
        },
        destructive: {
          DEFAULT: "hsl(var(--destructive))",
          foreground: "hsl(var(--destructive-foreground))",
        },
        muted: {
          DEFAULT: "hsl(var(--muted))",
          foreground: "hsl(var(--muted-foreground))",
        },
        accent: {
          DEFAULT: "hsl(var(--accent))",
          foreground: "hsl(var(--accent-foreground))",
        },
        popover: {
          DEFAULT: "hsl(var(--popover))",
          foreground: "hsl(var(--popover-foreground))",
        },
        card: {
          DEFAULT: "hsl(var(--card))",
          foreground: "hsl(var(--card-foreground))",
        },
      },
      borderRadius: {
        lg: "var(--radius)",
        md: "calc(var(--radius) - 2px)",
        sm: "calc(var(--radius) - 4px)",
      },
      keyframes: {
        "accordion-down": {
          from: { height: "0" },
          to: { height: "var(--radix-accordion-content-height)" },
        },
        "accordion-up": {
          from: { height: "var(--radix-accordion-content-height)" },
          to: { height: "0" },
        },
      },
      animation: {
        "accordion-down": "accordion-down 0.2s ease-out",
        "accordion-up": "accordion-up 0.2s ease-out",
      },
    },
  },
  plugins: [require("tailwindcss-animate")],
} satisfies Config

export default config
\end{lstlisting}

\subsection{Composants UI Réutilisables}

\subsubsection{Composant Button}

\begin{lstlisting}[language=TypeScript, caption=Composant Button réutilisable]
import * as React from "react"
import { Slot } from "@radix-ui/react-slot"
import { cva, type VariantProps } from "class-variance-authority"
import { cn } from "@/lib/utils"

const buttonVariants = cva(
  "inline-flex items-center justify-center whitespace-nowrap rounded-md text-sm font-medium ring-offset-background transition-colors focus-visible:outline-none focus-visible:ring-2 focus-visible:ring-ring focus-visible:ring-offset-2 disabled:pointer-events-none disabled:opacity-50",
  {
    variants: {
      variant: {
        default: "bg-primary text-primary-foreground hover:bg-primary/90",
        destructive:
          "bg-destructive text-destructive-foreground hover:bg-destructive/90",
        outline:
          "border border-input bg-background hover:bg-accent hover:text-accent-foreground",
        secondary:
          "bg-secondary text-secondary-foreground hover:bg-secondary/80",
        ghost: "hover:bg-accent hover:text-accent-foreground",
        link: "text-primary underline-offset-4 hover:underline",
      },
      size: {
        default: "h-10 px-4 py-2",
        sm: "h-9 rounded-md px-3",
        lg: "h-11 rounded-md px-8",
        icon: "h-10 w-10",
      },
    },
    defaultVariants: {
      variant: "default",
      size: "default",
    },
  }
)

export interface ButtonProps
  extends React.ButtonHTMLAttributes<HTMLButtonElement>,
    VariantProps<typeof buttonVariants> {
  asChild?: boolean
}

const Button = React.forwardRef<HTMLButtonElement, ButtonProps>(
  ({ className, variant, size, asChild = false, ...props }, ref) => {
    const Comp = asChild ? Slot : "button"
    return (
      <Comp
        className={cn(buttonVariants({ variant, size, className }))}
        ref={ref}
        {...props}
      />
    )
  }
)
Button.displayName = "Button"

export { Button, buttonVariants }
\end{lstlisting}

\section{Développement Backend}

\subsection{Configuration DynamoDB}

\subsubsection{Initialisation du Client}

\begin{lstlisting}[language=TypeScript, caption=Configuration DynamoDB Client]
import { DynamoDBClient } from "@aws-sdk/client-dynamodb";
import { DynamoDBDocumentClient } from "@aws-sdk/lib-dynamodb";

const dynamoDBConfig = {
  region: process.env.AWS_REGION || "us-east-1",
  ...(process.env.DYNAMODB_ENDPOINT && {
    endpoint: process.env.DYNAMODB_ENDPOINT,
  }),
  credentials: {
    accessKeyId: process.env.AWS_ACCESS_KEY_ID || "",
    secretAccessKey: process.env.AWS_SECRET_ACCESS_KEY || "",
  },
};

const ddbClient = new DynamoDBClient(dynamoDBConfig);
const docClient = DynamoDBDocumentClient.from(ddbClient, {
  marshallOptions: {
    convertEmptyValues: false,
    removeUndefinedValues: true,
    convertClassInstanceToMap: false,
  },
  unmarshallOptions: {
    wrapNumbers: false,
  },
});

export { ddbClient, docClient };
\end{lstlisting}

\subsubsection{Repository Pattern}

\begin{lstlisting}[language=TypeScript, caption=User Repository Implementation]
import { PutCommand, GetCommand, QueryCommand, ScanCommand, DeleteCommand } from "@aws-sdk/lib-dynamodb";
import { docClient } from "@/lib/aws/dynamodb-client";
import { User, CreateUserData } from "@/types/user";

export class UserRepository {
  private tableName = process.env.DYNAMODB_USERS_TABLE || "users";

  async create(userData: CreateUserData): Promise<User> {
    const user: User = {
      id: crypto.randomUUID(),
      ...userData,
      createdAt: new Date().toISOString(),
      updatedAt: new Date().toISOString(),
    };

    const command = new PutCommand({
      TableName: this.tableName,
      Item: user,
      ConditionExpression: "attribute_not_exists(id)",
    });

    try {
      await docClient.send(command);
      return user;
    } catch (error) {
      if (error.name === "ConditionalCheckFailedException") {
        throw new Error("User with this ID already exists");
      }
      throw error;
    }
  }

  async findById(id: string): Promise<User | null> {
    const command = new GetCommand({
      TableName: this.tableName,
      Key: { id },
    });

    const result = await docClient.send(command);
    return result.Item as User || null;
  }

  async findByEmail(email: string): Promise<User | null> {
    const command = new QueryCommand({
      TableName: this.tableName,
      IndexName: "EmailIndex",
      KeyConditionExpression: "email = :email",
      ExpressionAttributeValues: {
        ":email": email,
      },
    });

    const result = await docClient.send(command);
    return result.Items?.[0] as User || null;
  }

  async findAll(limit?: number): Promise<User[]> {
    const command = new ScanCommand({
      TableName: this.tableName,
      ...(limit && { Limit: limit }),
    });

    const result = await docClient.send(command);
    return result.Items as User[] || [];
  }

  async update(id: string, updates: Partial<User>): Promise<User | null> {
    const existingUser = await this.findById(id);
    if (!existingUser) {
      return null;
    }

    const updatedUser = {
      ...existingUser,
      ...updates,
      updatedAt: new Date().toISOString(),
    };

    const command = new PutCommand({
      TableName: this.tableName,
      Item: updatedUser,
    });

    await docClient.send(command);
    return updatedUser;
  }

  async delete(id: string): Promise<boolean> {
    const command = new DeleteCommand({
      TableName: this.tableName,
      Key: { id },
    });

    try {
      await docClient.send(command);
      return true;
    } catch (error) {
      console.error("Error deleting user:", error);
      return false;
    }
  }
}
\end{lstlisting}

\subsection{Gestion des Erreurs}

\subsubsection{Middleware de Gestion d'Erreurs}

\begin{lstlisting}[language=TypeScript, caption=Error Handling Middleware]
import { NextResponse } from "next/server";

export class AppError extends Error {
  public statusCode: number;
  public isOperational: boolean;

  constructor(message: string, statusCode: number) {
    super(message);
    this.statusCode = statusCode;
    this.isOperational = true;

    Error.captureStackTrace(this, this.constructor);
  }
}

export class ValidationError extends AppError {
  constructor(message: string) {
    super(message, 400);
  }
}

export class NotFoundError extends AppError {
  constructor(resource: string) {
    super(`${resource} not found`, 404);
  }
}

export class ConflictError extends AppError {
  constructor(message: string) {
    super(message, 409);
  }
}

export function handleError(error: unknown): NextResponse {
  console.error("API Error:", error);

  if (error instanceof AppError) {
    return NextResponse.json(
      {
        error: error.message,
        statusCode: error.statusCode,
      },
      { status: error.statusCode }
    );
  }

  if (error instanceof Error) {
    // DynamoDB specific errors
    if (error.name === "ValidationException") {
      return NextResponse.json(
        { error: "Invalid request data" },
        { status: 400 }
      );
    }

    if (error.name === "ResourceNotFoundException") {
      return NextResponse.json(
        { error: "Resource not found" },
        { status: 404 }
      );
    }

    if (error.name === "ConditionalCheckFailedException") {
      return NextResponse.json(
        { error: "Resource already exists" },
        { status: 409 }
      );
    }
  }

  // Generic server error
  return NextResponse.json(
    { error: "Internal server error" },
    { status: 500 }
  );
}
\end{lstlisting}

\section{Validation des Données}

\subsection{Schémas de Validation}

\subsubsection{Validation avec Zod}

\begin{lstlisting}[language=TypeScript, caption=Schémas de validation Zod]
import { z } from "zod";

export const createUserSchema = z.object({
  name: z.string()
    .min(2, "Name must be at least 2 characters")
    .max(100, "Name must not exceed 100 characters")
    .regex(/^[a-zA-Z\s]+$/, "Name can only contain letters and spaces"),
  
  email: z.string()
    .email("Please enter a valid email address")
    .max(255, "Email must not exceed 255 characters"),
  
  profilePicture: z.instanceof(File, "Profile picture is required")
    .refine((file) => file.size <= 5 * 1024 * 1024, "File size must be less than 5MB")
    .refine(
      (file) => ["image/jpeg", "image/png", "image/webp"].includes(file.type),
      "Only JPEG, PNG, and WebP images are allowed"
    ),
});

export const updateUserSchema = z.object({
  name: z.string()
    .min(2, "Name must be at least 2 characters")
    .max(100, "Name must not exceed 100 characters")
    .regex(/^[a-zA-Z\s]+$/, "Name can only contain letters and spaces")
    .optional(),
  
  email: z.string()
    .email("Please enter a valid email address")
    .max(255, "Email must not exceed 255 characters")
    .optional(),
  
  profilePicture: z.instanceof(File)
    .refine((file) => file.size <= 5 * 1024 * 1024, "File size must be less than 5MB")
    .refine(
      (file) => ["image/jpeg", "image/png", "image/webp"].includes(file.type),
      "Only JPEG, PNG, and WebP images are allowed"
    )
    .optional(),
});

export const userIdSchema = z.object({
  id: z.string().uuid("Invalid user ID format"),
});

export type CreateUserInput = z.infer<typeof createUserSchema>;
export type UpdateUserInput = z.infer<typeof updateUserSchema>;
export type UserIdInput = z.infer<typeof userIdSchema>;
\end{lstlisting}

\subsubsection{Middleware de Validation}

\begin{lstlisting}[language=TypeScript, caption=Validation Middleware]
import { NextRequest } from "next/server";
import { z } from "zod";
import { ValidationError } from "@/lib/errors";

export async function validateRequest<T>(
  request: NextRequest,
  schema: z.ZodSchema<T>
): Promise<T> {
  try {
    let data: any;

    const contentType = request.headers.get("content-type");
    
    if (contentType?.includes("application/json")) {
      data = await request.json();
    } else if (contentType?.includes("multipart/form-data")) {
      const formData = await request.formData();
      data = Object.fromEntries(formData.entries());
    } else {
      throw new ValidationError("Unsupported content type");
    }

    return schema.parse(data);
  } catch (error) {
    if (error instanceof z.ZodError) {
      const errorMessages = error.errors.map(err => 
        `${err.path.join('.')}: ${err.message}`
      ).join(', ');
      
      throw new ValidationError(`Validation failed: ${errorMessages}`);
    }
    
    throw error;
  }
}

export function validateParams<T>(
  params: any,
  schema: z.ZodSchema<T>
): T {
  try {
    return schema.parse(params);
  } catch (error) {
    if (error instanceof z.ZodError) {
      const errorMessages = error.errors.map(err => 
        `${err.path.join('.')}: ${err.message}`
      ).join(', ');
      
      throw new ValidationError(`Parameter validation failed: ${errorMessages}`);
    }
    
    throw error;
  }
}
\end{lstlisting}

\section{Intégration AWS S3}

\subsection{Service de Gestion de Fichiers}

\subsubsection{Configuration S3}

\begin{lstlisting}[language=TypeScript, caption=Configuration S3 Client]
import { S3Client, PutObjectCommand, DeleteObjectCommand } from "@aws-sdk/client-s3";
import { getSignedUrl } from "@aws-sdk/s3-request-presigner";
import crypto from "crypto";
import path from "path";

export class S3Service {
  private s3Client: S3Client;
  private bucketName: string;
  private region: string;

  constructor() {
    this.bucketName = process.env.AWS_S3_BUCKET_NAME!;
    this.region = process.env.AWS_REGION || "us-east-1";
    
    this.s3Client = new S3Client({
      region: this.region,
      credentials: {
        accessKeyId: process.env.AWS_ACCESS_KEY_ID!,
        secretAccessKey: process.env.AWS_SECRET_ACCESS_KEY!,
      },
    });
  }

  async uploadFile(file: File, folder: string = "uploads"): Promise<string> {
    const fileBuffer = Buffer.from(await file.arrayBuffer());
    const fileName = this.generateFileName(file.name);
    const key = `${folder}/${fileName}`;

    const command = new PutObjectCommand({
      Bucket: this.bucketName,
      Key: key,
      Body: fileBuffer,
      ContentType: file.type,
      Metadata: {
        originalName: file.name,
        uploadedAt: new Date().toISOString(),
      },
    });

    try {
      await this.s3Client.send(command);
      return this.getPublicUrl(key);
    } catch (error) {
      console.error("Error uploading file to S3:", error);
      throw new Error("Failed to upload file");
    }
  }

  async deleteFile(key: string): Promise<boolean> {
    const command = new DeleteObjectCommand({
      Bucket: this.bucketName,
      Key: key,
    });

    try {
      await this.s3Client.send(command);
      return true;
    } catch (error) {
      console.error("Error deleting file from S3:", error);
      return false;
    }
  }

  async getSignedUrl(key: string, expiresIn: number = 3600): Promise<string> {
    const command = new PutObjectCommand({
      Bucket: this.bucketName,
      Key: key,
    });

    return await getSignedUrl(this.s3Client, command, { expiresIn });
  }

  private generateFileName(originalName: string): string {
    const timestamp = Date.now();
    const randomString = crypto.randomBytes(8).toString("hex");
    const extension = path.extname(originalName);
    const baseName = path.basename(originalName, extension);
    const safeName = baseName.replace(/[^a-zA-Z0-9]/g, "-");
    
    return `${timestamp}-${randomString}-${safeName}${extension}`;
  }

  private getPublicUrl(key: string): string {
    return `https://${this.bucketName}.s3.${this.region}.amazonaws.com/${key}`;
  }
}
\end{lstlisting}

\section{Optimisations de Performance}

\subsection{Lazy Loading et Code Splitting}

\subsubsection{Dynamic Imports}

\begin{lstlisting}[language=TypeScript, caption=Implémentation du Lazy Loading]
import dynamic from "next/dynamic";
import { Suspense } from "react";
import { Skeleton } from "@/components/ui/skeleton";

// Lazy loading des composants lourds
const UserChart = dynamic(() => import("@/components/user-chart"), {
  loading: () => <Skeleton className="h-[300px] w-full" />,
  ssr: false,
});

const DataTable = dynamic(() => import("@/components/data-table"), {
  loading: () => (
    <div className="space-y-2">
      {Array.from({ length: 5 }).map((_, i) => (
        <Skeleton key={i} className="h-12 w-full" />
      ))}
    </div>
  ),
});

// Composant avec Suspense
export function Dashboard() {
  return (
    <div className="space-y-6">
      <Suspense fallback={<Skeleton className="h-[300px] w-full" />}>
        <UserChart />
      </Suspense>
      
      <Suspense fallback={<div>Loading table...</div>}>
        <DataTable />
      </Suspense>
    </div>
  );
}
\end{lstlisting}

\subsection{Caching Strategy}

\subsubsection{SWR Configuration}

\begin{lstlisting}[language=TypeScript, caption=Configuration SWR pour le cache]
import useSWR, { SWRConfig } from "swr";
import { fetcher } from "@/lib/api";

// Configuration globale SWR
export function SWRProvider({ children }: { children: React.ReactNode }) {
  return (
    <SWRConfig
      value={{
        fetcher,
        refreshInterval: 30000, // Refresh every 30 seconds
        revalidateOnFocus: true,
        revalidateOnReconnect: true,
        dedupingInterval: 5000,
        errorRetryCount: 3,
        errorRetryInterval: 1000,
        onError: (error) => {
          console.error("SWR Error:", error);
        },
      }}
    >
      {children}
    </SWRConfig>
  );
}

// Hook personnalisé pour les utilisateurs
export function useUsers() {
  const { data, error, mutate, isLoading } = useSWR<User[]>("/api/users");

  const createUser = async (userData: CreateUserInput) => {
    const optimisticData = [...(data || []), { ...userData, id: "temp" }];
    
    const options = {
      optimisticData,
      rollbackOnError: true,
      populateCache: true,
      revalidate: false,
    };

    return mutate(
      async () => {
        const response = await fetch("/api/users", {
          method: "POST",
          body: JSON.stringify(userData),
        });
        
        if (!response.ok) {
          throw new Error("Failed to create user");
        }
        
        return response.json();
      },
      options
    );
  };

  return {
    users: data,
    isLoading,
    error,
    createUser,
    mutate,
  };
}
\end{lstlisting}

\section{Sécurité et Authentification}

\subsection{CSRF Protection}

\subsubsection{CSRF Token Implementation}

\begin{lstlisting}[language=TypeScript, caption=Protection CSRF]
import { NextRequest, NextResponse } from "next/server";
import crypto from "crypto";

const CSRF_SECRET = process.env.CSRF_SECRET || "default-secret";

export function generateCSRFToken(): string {
  return crypto.randomBytes(32).toString("hex");
}

export function verifyCSRFToken(token: string, sessionToken: string): boolean {
  const expected = crypto
    .createHmac("sha256", CSRF_SECRET)
    .update(sessionToken)
    .digest("hex");
  
  return crypto.timingSafeEqual(Buffer.from(token), Buffer.from(expected));
}

export function csrfMiddleware(request: NextRequest): NextResponse | null {
  if (request.method === "GET" || request.method === "HEAD") {
    return null; // Skip CSRF for safe methods
  }

  const csrfToken = request.headers.get("x-csrf-token");
  const sessionToken = request.headers.get("x-session-token");

  if (!csrfToken || !sessionToken) {
    return NextResponse.json(
      { error: "CSRF token required" },
      { status: 403 }
    );
  }

  if (!verifyCSRFToken(csrfToken, sessionToken)) {
    return NextResponse.json(
      { error: "Invalid CSRF token" },
      { status: 403 }
    );
  }

  return null; // Continue processing
}
\end{lstlisting}

Cette implémentation détaillée illustre les décisions techniques et les bonnes pratiques adoptées durant le développement d'AWS Next Express, garantissant un code maintenable, performant et sécurisé. 